% CS615A Aspects of System Administration
% Author: Jan Schaumann <jschauma@netmeister.org>
% $Id: slides.tex,v 1.12 2006/02/27 02:53:29 jschauma Exp $
\special{! TeXDict begin /landplus90{true}store end }

\documentclass[xga]{xdvislides}
\usepackage[landscape]{geometry}
\usepackage{graphics}
\usepackage{graphicx}
\usepackage{colordvi}

\begin{document}
\setfontphv

%%% Headers and footers
\lhead{\slidetitle}                               % default:\lhead{\slidetitle}
\chead{CS615 - Aspects of System Administration}% default:\chead{\relax}
\rhead{Slide \thepage}                       % default:\rhead{\sectiontitle}
\lfoot{\Gray{Automating Administrative Tasks}}% default:\lfoot{\slideauthor}
\cfoot{\relax}                               % default:\cfoot{\relax}
\rfoot{\Gray{\today}}

\vspace*{\fill}
\begin{center}
	\Hugesize
		CS615 - Aspects of System Administration\\ [1em]
		Automating Administrative Tasks / Shell Scripting\\ [1em]

	\hspace*{5mm}\blueline\\ [1em]
	\Normalsize
		Department of Computer Science\\
		Stevens Institute of Technology\\
		Jan Schaumann\\
		\verb+jschauma@stevens.edu+\\
		\verb+http://www.cs.stevens.edu/~jschauma/615/+
\end{center}
\vspace*{\fill}

%\subsection{Topics Covered}
%Overview:
%\begin{itemize}
%	\item why use automation
%	\item when to use automation
%	\item what tools to use
%\end{itemize}
%
%\subsection{Topics Covered}
%Overview:
%\begin{itemize}
%	\item why use automation
%	\item when to use automation
%	\item what tools to use
%\end{itemize}
%\addvspace{.5in}
%In more detail:
%\begin{itemize}
%	\item Shell Basics
%\end{itemize}
%
%\subsection{Topics Covered}
%Overview:
%\begin{itemize}
%	\item why use automation
%	\item when to use automation
%	\item what tools to use
%\end{itemize}
%\addvspace{.5in}
%In more detail:
%\begin{itemize}
%	\item Shell Basics
%	\item Regular Expressions
%\end{itemize}
%
%\subsection{Topics Covered}
%Overview:
%\begin{itemize}
%	\item why use automation
%	\item when to use automation
%	\item what tools to use
%\end{itemize}
%\addvspace{.5in}
%In more detail:
%\begin{itemize}
%	\item Shell Basics
%	\item Regular Expressions
%	\item Shell Programming
%\end{itemize}
%
%\subsection{Topics Covered}
%Overview:
%\begin{itemize}
%	\item why use automation
%	\item when to use automation
%	\item what tools to use
%\end{itemize}
%\addvspace{.5in}
%In more detail:
%\begin{itemize}
%	\item Shell Basics
%	\item Regular Expressions
%	\item Shell Programming
%	\item Perl, Python, ...
%\end{itemize}
%
%\subsection{Topics Covered}
%Overview:
%\begin{itemize}
%	\item why use automation
%	\item when to use automation
%	\item what tools to use
%\end{itemize}
%\addvspace{.5in}
%In more detail:
%\begin{itemize}
%	\item Shell Basics
%	\item Regular Expressions
%	\item Shell Programming
%	\item Perl, Python, ...
%	\item when only C will do
%\end{itemize}
%
%\subsection{Why?  When?  How?}
%Automating tasks has several benefits:
%\begin{itemize}
%	\item greater reliability
%\end{itemize}
%
%\subsection{Why?  When?  How?}
%Automating tasks has several benefits:
%\begin{itemize}
%	\item greater reliability
%	\item guaranteed regularity
%\end{itemize}
%
%\subsection{Why?  When?  How?}
%Automating tasks has several benefits:
%\begin{itemize}
%	\item greater reliability
%	\item guaranteed regularity
%	\item enhanced system efficiency
%\end{itemize}
%
%\subsection{Why?  When?  How?}
%Automating tasks has several benefits:
%\begin{itemize}
%	\item greater reliability
%	\item guaranteed regularity
%	\item enhanced system efficiency
%\end{itemize}
%\vspace{.5in}
%Automate tasks for
%\begin{itemize}
%	\item yourself
%\end{itemize}
%
\subsection{The 7 cardinal virtues of System Administrators}
\vspace*{\fill}
\begin{center}
	\includegraphics[scale=1.2]{pics/flexibility.eps}
\end{center}
\vspace*{\fill}

\subsection{The 7 cardinal virtues of System Administrators}
\vspace*{\fill}
\begin{center}
	\includegraphics[scale=0.8]{pics/ingenuity.eps} \\
\end{center}
\vspace*{\fill}

%\subsection{The 7 cardinal virtues of System Administrators}
%\vspace*{\fill}
%\begin{center}
%	\includegraphics[scale=0.55,angle=-90]{pics/ingenuity-can-should.eps}
%\end{center}
%\vspace*{\fill}
%
\subsection{The 7 cardinal virtues of System Administrators}
\vspace*{\fill}
\begin{center}
	\includegraphics[scale=0.2]{pics/waldo.eps}
\end{center}
\vspace*{\fill}

\subsection{The 7 cardinal virtues of System Administrators}
\vspace*{\fill}
\begin{center}
	\includegraphics[scale=4]{pics/DinnerForOne.eps}
\end{center}
\vspace*{\fill}

\subsection{The 7 cardinal virtues of System Administrators}
\vspace*{\fill}
\begin{center}
	\includegraphics[scale=0.8]{pics/try_try_again.eps}
\end{center}
\vspace*{\fill}

\subsection{The 7 cardinal virtues of System Administrators}
\vspace*{\fill}
\begin{center}
	\includegraphics[scale=5]{pics/patience.eps}
\end{center}
\vspace*{\fill}

\subsection{The 7 cardinal virtues of System Administrators}
\vspace*{\fill}
\begin{center}
	\includegraphics[scale=1.0]{pics/height_of_laziness.eps}
\end{center}
\vspace*{\fill}


\subsection{The 7 cardinal virtues of System Administrators}
\begin{itemize}
	\item flexibility
	\item ingenuity
	\item attention to detail
	\item adherence to routine
	\item persistence
	\item patience
	\item laziness
\end{itemize}

\subsection{The 7 cardinal virtues of System Administrators}
\begin{itemize}
	\item flexibility
	\item ingenuity
	\item attention to detail
	\item adherence to routine
	\item persistence
	\item patience
	\item laziness
\end{itemize}
\addvspace{.5in}

In addition:
\begin{itemize}
	\item impatience
	\item hubris
\end{itemize}

\subsection{Why?  When?  How?}
Automating tasks has several benefits:
\begin{itemize}
	\item greater reliability
	\item guaranteed regularity
	\item enhanced system efficiency
\end{itemize}
\vspace{.5in}
Automate tasks for
\begin{itemize}
	\item yourself
	\item other system administrators
\end{itemize}

\subsection{Why?  When?  How?}
Automating tasks has several benefits:
\begin{itemize}
	\item greater reliability
	\item guaranteed regularity
	\item enhanced system efficiency
\end{itemize}
\vspace{.5in}
Automate tasks for
\begin{itemize}
	\item yourself
	\item other system administrators
	\item all users
\end{itemize}


\subsection{Tools}
\vspace*{\fill}
\begin{center}
	\includegraphics[scale=1.4]{pics/tools.eps}
\end{center}
\vspace*{\fill}

\subsection{The right tool?}
\vspace*{\fill}
\begin{center}
	\includegraphics[scale=0.6]{pics/hammer.eps}
\end{center}
\vspace*{\fill}

\subsection{The right tool?}
\vspace*{\fill}
\begin{center}
	\includegraphics[scale=2.5]{pics/nails.eps}
\end{center}
\vspace*{\fill}

\subsection{The right tool?}
\vspace*{\fill}
\begin{center}
	\includegraphics[scale=3]{pics/hammer-screw.eps}
\end{center}
\vspace*{\fill}

\subsection{A better hammer!}
\vspace*{\fill}
\begin{center}
	\includegraphics[scale=3]{pics/mchammer.eps}
\end{center}
\vspace*{\fill}


\subsection{Approaching Automation}
Three basic categories:
\\

\begin{itemize}
	\item scripting
	\item programming
	\item software development
\end{itemize}

\subsection{Approaching Automation}
Three basic categories:
\\
\begin{itemize}
	\item scripting
\end{itemize}
\vspace*{\fill}
\begin{center}
	\includegraphics[scale=0.5]{pics/hammer.eps}
\end{center}
\vspace*{\fill}


\subsection{Approaching Automation}
Three basic categories:
\\

\begin{itemize}
	\item scripting
		\begin{itemize}
			\item automating {\em very} simple tasks
		\end{itemize}
\end{itemize}

\subsection{Approaching Automation}
Three basic categories:
\\

\begin{itemize}
	\item scripting
		\begin{itemize}
			\item automating {\em very} simple tasks
			\item customization of user environment
		\end{itemize}
\end{itemize}

\subsection{Approaching Automation}
Three basic categories:
\\

\begin{itemize}
	\item scripting
		\begin{itemize}
			\item automating {\em very} simple tasks
			\item customization of user environment
			\item often only suitable for one individual user
		\end{itemize}
\end{itemize}

\subsection{Approaching Automation}
Three basic categories:
\\

\begin{itemize}
	\item scripting
		\begin{itemize}
			\item automating {\em very} simple tasks
			\item customization of user environment
			\item often only suitable for one individual user
			\item usually eventually evolves into larger programs
		\end{itemize}
\end{itemize}


\subsection{Approaching Automation}
Three basic categories:
\\

\begin{itemize}
	\item programming
\end{itemize}
\vspace*{\fill}
\begin{center}
	\includegraphics[scale=0.6]{pics/mchammer2.eps}
\end{center}
\vspace*{\fill}

\subsection{Approaching Automation}
Three basic categories:
\\

\begin{itemize}
	\item programming
\end{itemize}
\vspace*{\fill}
\begin{center}
	\includegraphics[scale=0.8]{pics/power-tools.eps}
\end{center}
\vspace*{\fill}



\subsection{Approaching Automation}
Three basic categories:
\\

\begin{itemize}
	\item programming
		\begin{itemize}
			\item suitable for simple to moderately complex tasks
		\end{itemize}
\end{itemize}

\subsection{Approaching Automation}
Three basic categories:
\\

\begin{itemize}
	\item programming
		\begin{itemize}
			\item suitable for simple to moderately complex tasks
			\item results frequently used by a small base of users
		\end{itemize}
\end{itemize}

\subsection{Approaching Automation}
Three basic categories:
\\

\begin{itemize}
	\item programming
		\begin{itemize}
			\item suitable for simple to moderately complex tasks
			\item results frequently used by a small base of users
			\item uses basic framework or common toolkits
		\end{itemize}
\end{itemize}

\subsection{Approaching Automation}
Three basic categories:
\\

\begin{itemize}
	\item programming
		\begin{itemize}
			\item suitable for simple to moderately complex tasks
			\item results frequently used by a small base of users
			\item uses basic framework or common toolkits
			\item provides consistent interface
		\end{itemize}
\end{itemize}

\subsection{Approaching Automation}
Three basic categories:
\\

\begin{itemize}
	\item programming
		\begin{itemize}
			\item suitable for simple to moderately complex tasks
			\item results frequently used by a small base of users
			\item uses basic framework or common toolkits
			\item provides consistent interface
			\item may evolve into full product
		\end{itemize}
\end{itemize}

\subsection{Approaching Automation}
Three basic categories:
\\

\begin{itemize}
	\item software development
\end{itemize}
\vspace*{\fill}
\begin{center}
	\includegraphics[scale=0.25]{pics/big-tool.eps}
\end{center}
\vspace*{\fill}


\subsection{Approaching Automation}
Three basic categories:
\\

\begin{itemize}
	\item software development
		\begin{itemize}
			\item required for any reasonably complex task
		\end{itemize}
\end{itemize}

\subsection{Approaching Automation}
Three basic categories:
\\

\begin{itemize}
	\item software development
		\begin{itemize}
			\item required for any reasonably complex task
			\item uses formal software engineering approach (measurable goals,
				requirements, specifications, ...)
		\end{itemize}
\end{itemize}

\subsection{Approaching Automation}
Three basic categories:
\\

\begin{itemize}
	\item software development
		\begin{itemize}
			\item required for any reasonably complex task
			\item uses formal software engineering approach (measurable goals,
				requirements, specifications, ...)
			\item may evolve from previous prototypes
		\end{itemize}
\end{itemize}


\subsection{Approaching Automation}
Three basic categories:
\\

\begin{itemize}
	\item software development
		\begin{itemize}
			\item required for any reasonably complex task
			\item uses formal software engineering approach (measurable goals,
				requirements, specifications, ...)
			\item may evolve from previous prototypes
			\item requires ongoing continous maintenance / development efforts
		\end{itemize}
\end{itemize}


\newpage
\vspace*{\fill}
\begin{center}
	\Hugesize
		Jan's Words of Wisdom \\ [1em]
	\hspace*{5mm}
	\blueline\\
	\hspace*{5mm}\\
		Tuesday, February 21st, 2012
\end{center}
\vspace*{\fill}

\subsection{Words of Wisdom}
\\

\newcommand{\gargantuan}{\fontsize{70}{75}\selectfont}
\gargantuan
\begin{center}
Anything you do more than once is worth automating.
\end{center}
\Normalsize

\subsection{Words of Wisdom}
\vspace*{\fill}
\begin{center}
	\includegraphics[scale=3.0]{pics/growmoney.eps}
\end{center}
\vspace*{\fill}

\subsection{Words of Wisdom}
\vspace*{\fill}
\begin{center}
	\includegraphics[scale=3.5]{pics/money-tree.eps}
\end{center}
\vspace*{\fill}

\subsection{Good tools exhibit our cardinal virtues}
Our software defines, exhibits or provides:
\begin{itemize}
	\item flexibility
	\item ingenuity
	\item attention to detail
	\item adherence to routine
	\item persistence
	\item patience
	\item laziness
\end{itemize}

\subsection{Scheduling programs}

\begin{itemize}
	\item repeatedly $\rightarrow$ \verb+cron(8)+, \verb+crontab(1)+ \\
		\verb,30 5 * * 6 find /tmp -type f -mtime +7 -atime +7 -exec rm -f '{}' ';',
	\item once $\rightarrow$ \verb+at(1)+ \\
		\verb,at -f script.sh 1am tomorrow,
	\item based on system load $\rightarrow$ \verb+batch(1)+ \\
		\verb,batch -f script.sh 4pm + 3 days,
\end{itemize}


\newpage
\vspace*{\fill}
\begin{center}
    \Hugesize
        It's coding time! \\ [1em]
    \hspace*{5mm}
    \blueline\\
    \hspace*{5mm}\\
	Get out your laptops...
\end{center}
\vspace*{\fill}

\subsection{Coding time!}
\vspace*{\fill}
\begin{center}
	\includegraphics[scale=2]{pics/donquixote.eps}
\end{center}
\vspace*{\fill}

\subsection{Coding time!}
Tell me how many words are there in Miguel de Cervantes's ``Don Quixote''.

\subsection{Coding time!}
Tell me how many words are there in Miguel de Cervantes's ``Don Quixote''.
\\

\begin{verbatim}
curl http://www.gutenberg.org/cache/epub/996/pg996.txt | wc
\end{verbatim}

\subsection{Coding time!}
Tell me how many words are there in Miguel de Cervantes's ``Don Quixote''.
\\

\begin{verbatim}
curl http://www.gutenberg.org/cache/epub/996/pg996.txt | wc
\end{verbatim}
\vspace{.5in}

Ok, which word is the most common in this text?


\subsection{Coding time!}
Write a shell script that:
\begin{itemize}
	\item reads input from {\tt stdin}
	\item counts the frequency of each word
	\item prints the frequency followed by the word to stdout
\end{itemize}
\vspace{.5in}
You have 20 minutes... \\

\vspace{.5in}
(If you finish before 20 minutes are up, write a script to count the
frequency of characters, too.) \\

\vspace{.5in}
(If you finish {\em this} before 20 minutes are up, write a script
to print the frequency of words by word-length.)

\subsection{Coding time!}
One solution:
\begin{verbatim}
# translate spaces into newlines
tr '[:space:]' '\n' |                    \
        # count unique occurrences
        sort | uniq -c
\end{verbatim}

\subsection{Coding time!}
One solution:
\begin{verbatim}
# translate spaces into newlines
tr '[:space:]' '\n' |                    \
        # ignore punctuation
        tr -d '[:punct:]' |              \
        # count unique occurrences
        sort | uniq -c
\end{verbatim}

\subsection{Coding time!}
One solution:
\begin{verbatim}
# translate spaces into newlines
tr '[:space:]' '\n' |                    \
        # ignore punctuation
        tr -d '[:punct:]' |              \
        # treat upper and lower cases the same
        tr '[:upper:]' '[:lower:]' |     \
        # count unique occurrences
        sort | uniq -c
\end{verbatim}

\subsection{Coding time!}
One solution:
\begin{verbatim}
# ignore punctuation
tr '[:punct:]' ' ' |                     \
        # translate spaces into newlines
        tr '[:space:]' '\n' |            \
        # treat upper and lower cases the same
        tr '[:upper:]' '[:lower:]' |     \
        # count unique occurrences
        sort | uniq -c
\end{verbatim}

\subsection{Coding time!}
One solution:
\begin{verbatim}
# ignore punctuation
tr '[:punct:]' ' ' |                     \
        # translate spaces into newlines
        tr '[:space:]' '\n' |            \
        # treat upper and lower cases the same
        tr '[:upper:]' '[:lower:]' |     \
        # ignore numbers and other non-word words
        grep -v '[^a-z]' |               \
        # count unique occurrences
        sort | uniq -c
\end{verbatim}

\subsection{Coding time!}
One solution:
\begin{verbatim}
# ignore punctuation
tr '[:punct:]' ' ' |                     \
        # translate spaces into newlines
        tr '[:space:]' '\n' |            \
        # treat upper and lower cases the same
        tr '[:upper:]' '[:lower:]' |     \
        # ignore numbers and other non-word words
        egrep '^[a-z]+$' |               \
        # count unique occurrences
        sort | uniq -c
\end{verbatim}

\subsection{Coding time!}
One solution:
\begin{verbatim}
# ignore punctuation
tr '[:punct:]' ' ' |                     \
        # translate spaces into newlines
        tr '[:space:]' '\n' |            \
        # treat upper and lower cases the same
        tr '[:upper:]' '[:lower:]' |     \
        # ignore numbers and other non-word words
        egrep '^[a-z]+$' |               \
        # count unique occurrences
        sort | uniq -c
\end{verbatim}
\vspace{.5in}
Character frequency:
\begin{verbatim}
fold -w1 | tr '[:upper:]' '[:lower:]' | egrep '^[a-z]+$' | sort | uniq -c
\end{verbatim}

\subsection{Coding time!}
One solution:
\begin{verbatim}
# ignore punctuation
tr '[:punct:]' ' ' |                     \
        # translate spaces into newlines
        tr '[:space:]' '\n' |            \
        # treat upper and lower cases the same
        tr '[:upper:]' '[:lower:]' |     \
        # ignore numbers and other non-word words
        egrep '^[a-z]+$' |               \
        # count unique occurrences
        sort | uniq -c
\end{verbatim}
Character frequency:
\begin{verbatim}
fold -w1 | tr '[:upper:]' '[:lower:]' | egrep '^[a-z]+$' | sort | uniq -c
\end{verbatim}
Word length:
\begin{verbatim}
tr '[:punct:]' ' ' | tr '[:space:]' '\n' | tr '[:upper:]' '[:lower:]' | \
        egrep '^[a-z]+$' | awk '{ print length(); }' | sort -n | uniq -c
\end{verbatim}


\subsection{Shell Essentials}
Essential Tools:
\begin{itemize}
	\item pipes
	\item ls(1), find(1)
	\item grep(1)
	\item awk(1)
	\item sed(1)
	\item tr(1)
	\item regular expressions
	\item ssh(1)/scp(1)
	\item expect(1)
\end{itemize}

\subsection{awk(1)}
{\tt examples/runcmd.sh}

{\tt examples/pkg\_ids}

\subsection{sed(1)}

\begin{verbatim}
sed -e 's/\([^:]*\).*:\(.*$\)/\1 -> \2/' /etc/passwd

http://sed.sourceforge.net/sed1line.txt

examples/hanoi.sed
\end{verbatim}


\subsection{Shell Essentials}
What does a shell do?
\begin{itemize}
	\item ``simple'' interactive program
	\item fork-exec commands
	\item job-control
	\item input-output redirection
\end{itemize}

\subsection{Writing Shell Scripts: Startup and Environment}
\begin{itemize}
	\item when the shell starts, it will read various startup files as
		well as initialize some variables
\end{itemize}

\subsection{Writing Shell Scripts: Startup and Environment}
\begin{itemize}
	\item when the shell starts, it will read various startup files as
		well as initialize some variables
	\item difference between use as an interactive (or login) shell and a
		command-interpreter/programming language
\end{itemize}

\subsection{Writing Shell Scripts: Startup and Environment}
\begin{itemize}
	\item when the shell starts, it will read various startup files as
		well as initialize some variables
	\item difference between use as an interactive (or login) shell and a
		command-interpreter/programming language
	\item remember that each process inherits the process environment
		from its parent
\end{itemize}

\subsection{Writing Shell Scripts: Startup and Environment}
\begin{itemize}
	\item when the shell starts, it will read various startup files as
		well as initialize some variables
	\item difference between use as an interactive (or login) shell and a
		command-interpreter/programming language
	\item remember that each process inherits the process environment
		from its parent
	\item \verb+export+ed variables vs local variables
\end{itemize}

\subsection{Writing Shell Scripts: Startup and Environment}
\begin{itemize}
	\item when the shell starts, it will read various startup files as
		well as initialize some variables
	\item difference between use as an interactive (or login) shell and a
		command-interpreter/programming language
	\item remember that each process inherits the process environment
		from its parent
	\item \verb+export+ed variables vs local variables
	\item some variables influence or apply to many applications
\end{itemize}


\subsection{Writing Shell Scripts: Startup and Environment}
\begin{itemize}
	\item when the shell starts, it will read various startup files as
		well as initialize some variables
	\item difference between use as an interactive (or login) shell and a
		command-interpreter/programming language
	\item remember that each process inherits the process environment
		from its parent
	\item \verb+export+ed variables vs local variables
	\item some variables influence or apply to many applications
\end{itemize}

\subsection{Writing Shell Scripts: Shell Variables}
\begin{itemize}
	\item {\tt \$*} and {\tt \$@}
	\item {\tt \$\#}, {\tt \$0} and {\tt \$1 ... \$N}
	\item {\tt \$?}
	\item {\tt \$\$}
	\item {\tt \$!}
\end{itemize}

\subsection{Writing Shell Scripts: Shell Variable Expansions}
\begin{itemize}
	\item {\tt \$\{var:-default\}}
	\item {\tt \$\{var:?"Message\}}
	\item {\tt \$\{var:+value\}}
	\item {\tt \$\{var\%suffix\}} and {\tt \$\{var\%\%suffix\}}
	\item {\tt \$\{var\#prefix\}} and {\tt \$\{var\#\#prefix\}}
\end{itemize}

\subsection{Writing Shell Scripts: ...and there's still more!}
\begin{itemize}
	\item {\tt \$(command)} or {\tt \`{}command\`{}}
	\item {\tt \$(( arithmetic expression ))}
	\item stdin, stdout, stderr
	\item \verb+>+, \verb+>>+, \verb+<+, \verb+<<+, \verb+|+
\end{itemize}
\vspace{.5in}
{\tt sh examples/shexamples}

\subsection{Writing Shell Scripts: Builtins, Functions, Job Control, Signals}
\begin{itemize}
	\item {\em builtins} are executed internally (ie without spawning a
		new process)
\end{itemize}

\subsection{Writing Shell Scripts: Builtins, Functions, Job Control, Signals}
\begin{itemize}
	\item {\em builtins} are executed internally (ie without spawning a
		new process)
	\item some {\em builtins} cannot be performed (efficiently) by
		separate processes, some can (and are)
\end{itemize}

\subsection{Writing Shell Scripts: Builtins, Functions, Job Control, Signals}
\begin{itemize}
	\item {\em builtins} are executed internally (ie without spawning a
		new process)
	\item some {\em builtins} cannot be performed (efficiently) by
		separate processes, some can (and are)
	\item {\em functions} are executed within the current shell
\end{itemize}

\subsection{Writing Shell Scripts: Builtins, Functions, Job Control, Signals}
\begin{itemize}
	\item {\em builtins} are executed internally (ie without spawning a
		new process)
	\item some {\em builtins} cannot be performed (efficiently) by
		separate processes, some can (and are)
	\item {\em functions} are executed within the current shell
	\item job control may work differently within {\em functions}
\end{itemize}

\subsection{Writing Shell Scripts: Builtins, Functions, Job Control, Signals}
\begin{itemize}
	\item {\em builtins} are executed internally (ie without spawning a
		new process)
	\item some {\em builtins} cannot be performed (efficiently) by
		separate processes, some can (and are)
	\item {\em functions} are executed within the current shell
	\item job control may work differently within {\em functions}
	\item {\em functions} operate on and influence the current environment
\end{itemize}

\subsection{Writing Shell Scripts: Builtins, Functions, Job Control, Signals}
\begin{itemize}
	\item {\em builtins} are executed internally (ie without spawning a
		new process)
	\item some {\em builtins} cannot be performed (efficiently) by
		separate processes, some can (and are)
	\item {\em functions} are executed within the current shell
	\item job control may work differently within {\em functions}
	\item {\em functions} operate on and influence the current environment
	\item shells can monitor and control {\em jobs}
\end{itemize}

\subsection{Writing Shell Scripts: Builtins, Functions, Job Control, Signals}
\begin{itemize}
	\item {\em builtins} are executed internally (ie without spawning a
		new process)
	\item some {\em builtins} cannot be performed (efficiently) by
		separate processes, some can (and are)
	\item {\em functions} are executed within the current shell
	\item job control may work differently within {\em functions}
	\item {\em functions} operate on and influence the current environment
	\item shells can monitor and control {\em jobs}
	\item some {\em signals} may be ignored or caught and handled, others
		may not be ignored
\end{itemize}

\subsection{Writing Shell Scripts: Builtins, Functions, Job Control, Signals}
\begin{itemize}
	\item {\em builtins} are executed internally (ie without spawning a
		new process)
	\item some {\em builtins} cannot be performed (efficiently) by
		separate processes, some can (and are)
	\item {\em functions} are executed within the current shell
	\item job control may work differently within {\em functions}
	\item {\em functions} operate on and influence the current environment
	\item shells can monitor and control {\em jobs}
	\item some {\em signals} may be ignored or caught and handled, others
		may not be ignored
\end{itemize}
\addvspace{.5in}
As you can tell, this whole thing looks like a full-blown programming language!



\subsection{Regular Expressions}
A regular expression is a pattern that describes a set of strings. \\

\begin{itemize}
	\item patterns can be
		\begin{itemize}
			\item single characters (\verb+a+)
			\item a bracket expression, such as
				\begin{itemize}
					\item a number of characters -- \verb+[aK2l,]+
					\item a range expression -- \verb+[a-z]+
					\item a negated bracket expression -- \verb+[^0-9]+
				\end{itemize}
			\item a character with a special meaning, such as
				\begin{itemize}
					\item \verb+.+ -- any single character
					\item \verb+^+ -- beginning of line
					\item \verb+$+ -- end of line
				\end{itemize}
			\item a combination of patterns
		\end{itemize}
\end{itemize}

\subsection{Regular Expressions}
\begin{itemize}
	\item patterns can be followed by qualifiers and quantifiers
		\begin{itemize}
			\item \verb+?+ -- the pattern is optional and matched at most once
			\item \verb+*+ -- the pattern will be matched zero or more times
			\item \verb|+| -- the pattern will be matched one or more times
			\item \verb+{n}+ -- the pattern is matched exactly \verb+n+ times
			\item \verb+{n,}+ -- the pattern is matched \verb+n+ or more times
			\item \verb+{n,m}+ -- the pattern is matched at least \verb+n+,
				but no more than \verb+m+ times
		\end{itemize}
	\item patterns can be logically grouped together \verb+(1[a-z]2|a[0-9]z)+
	\item matched patterns can be remembered and referenced lateron
\end{itemize}
\addvspace{.5in}
NB: different tools implement regular expressions somewhat differently

\subsection{Regular Expressions}
Example exercises:
\begin{itemize}
	\item extract all proper words from a document
	\item extract all URLs from a document
	\item check if a string is a valid IPv4 address
	\item check if a string is a valid IPv6 address
	\item check if a string is a valid date
\end{itemize}

\newpage
\vspace*{\fill}
\begin{center}
    \Hugesize
        Hooray! \\ [1em]
    \hspace*{5mm}
    \blueline\\
    \hspace*{5mm}\\
        5 Minute Break
\end{center}
\vspace*{\fill}
%
%\subsection{Scripting / interpreted Languages}
%\vspace*{\fill}
%\begin{center}
%	\includegraphics[scale=0.9]{pics/compiling.eps}
%	\\
%	\small \verb+http://xkcd.com/303/+
%\end{center}
%\vspace*{\fill}
%
%\subsection{Scripting / interpreted Languages}
%\\
%
%\Huge
%\begin{center}
%	Perl, Python, Ruby, \\
%	\addvspace{.5in}
%	PHP, Lisp, Smalltalk, Tcl, \\
%	\addvspace{.5in}
%	ECMA (JavaScript etc.), Lua
%\end{center}
%\Normalsize
%
%
%\subsection{Scripting / interpreted Languages}
%General advantages:
%\begin{itemize}
%	\item short development cycle
%	\item normally facilitates things like string manipulation,
%		arithmetic and more complex regular expressions
%	\item easily handles multiple file handles and other I/O
%	\item some security features
%	\item tens of thousands of special- and general-purpose modules
%		available
%\end{itemize}
%
%\subsection{Scripting / interpreted Languages}
%General disadvantages:
%\begin{itemize}
%	\item no one tool fits all purposes
%	\item tens of thousands of special- and general-purpose modules
%		available $\rightarrow$ lots of duplication, stale code,
%		questionable quality
%	\item security features frequently neglected or circumvented ("too
%		hard" or more precisely "inconvenient")
%	\item everybody has their particular favorite (and dislikes one or
%		the other)
%	\item interpreter not (necessarily) universally available /
%		installed
%\end{itemize}
%
%\subsection{When only C will do}
%\begin{itemize}
%	\item often faster / more efficient
%	\item inherent part of UNIX
%	\item scales better
%	\item enough rope to hang yourself
%	\item Face it:  you need to know C anyway
%\end{itemize}
%
%\subsection{User Interface}
%\\
%\vspace*{\fill}
%\begin{center}
%	\includegraphics[scale=3]{pics/switch.eps}
%\end{center}
%\vspace*{\fill}
%
%\subsection{Unix Philosophy}
%\\
%\vspace*{\fill}
%\begin{center}
%	\includegraphics[scale=1.5]{pics/pipe.eps}
%\end{center}
%\vspace*{\fill}
%
%\subsection{Unix Philosophy}
%\\
%\Huge
%\begin{center}
%	Do one thing and do it well.
%\end{center}
%\Normalsize
%
%\subsection{The KISS Principle}
%\\
%\vspace*{\fill}
%\begin{center}
%	\includegraphics[scale=0.3]{pics/kiss.eps}
%\end{center}
%\vspace*{\fill}
%
%\subsection{POLA}
%Principle of Least Astonishment
%\\
%\vspace*{\fill}
%\begin{center}
%	\includegraphics[scale=0.8]{pics/kinder-surprise.eps}
%\end{center}
%\vspace*{\fill}
%
%\subsection{Test Driven}
%\vspace*{\fill}
%\begin{center}
%	\includegraphics[scale=2.0]{pics/tdd.eps}
%\end{center}
%\vspace*{\fill}
%
%
%\subsection{The Zen of Python}
%\Huge
%\begin{center}
%Beautiful is better than ugly.
%\end{center}
%
%\subsection{The Zen of Python}
%\begin{center}
%Explicit is better than implicit.
%\end{center}
%
%\subsection{The Zen of Python}
%\begin{center}
%    Simple is better than complex.
%\vspace*{\fill}
%	\includegraphics[scale=0.8]{pics/complexity.eps}
%	\\
%	\small \verb+http://xkcd.com/399/+
%\end{center}
%\vspace*{\fill}
%\Huge
%
%\subsection{The Zen of Python}
%\begin{center}
%    Complex is better than complicated.
%\end{center}
%
%\subsection{The Zen of Python}
%\begin{center}
%    Flat is better than nested.
%\end{center}
%
%\subsection{The Zen of Python}
%\begin{center}
%    Sparse is better than dense.
%\end{center}
%
%\subsection{The Zen of Python}
%\begin{center}
%    Readability counts.
%\end{center}
%
%\subsection{The Zen of Python}
%\begin{center}
%    Special cases aren't special enough to break the rules.
%\end{center}
%
%\subsection{The Zen of Python}
%\begin{center}
%    Special cases aren't special enough to break the rules. \\
%\addvspace{.5in}
%    Although practicality beats purity.
%\end{center}
%
%\subsection{The Zen of Python}
%\\
%\begin{center}
%    Errors should never pass silently.
%\end{center}
%
%\subsection{The Zen of Python}
%\\
%\begin{center}
%    Errors should never pass silently. \\
%\addvspace{.2in}
%	\small
%	(That would be implicitly accepted failure.)
%\end{center}
%\Huge
%
%\subsection{The Zen of Python}
%\\
%\begin{center}
%    Errors should never pass silently. \\
%\addvspace{.2in}
%	\small
%	(That would be implicitly accepted failure.) \\
%\addvspace{.2in}
%	(You know what would be better than something {\em implicit}?)
%\end{center}
%
%\subsection{The Zen of Python}
%\\
%\begin{center}
%    Errors should never pass silently. \\
%\addvspace{.2in}
%	\small
%	(That would be implicitly accepted failure.) \\
%\addvspace{.2in}
%	(You know what would be better than something {\em implicit}?) \\
%\addvspace{.2in}
%	(Why, of course, something {\em explicit}!)
%\end{center}
%
%\subsection{The Zen of Python}
%\\
%\begin{center}
%    Errors should never pass silently. \\
%\addvspace{.5in}
%    Unless explicitly silenced.
%\end{center}
%
%\subsection{The Zen of Python}
%\begin{center}
%    In the face of ambiguity, refuse the temptation to guess.
%\end{center}
%
%\subsection{The Zen of Python}
%\begin{center}
%    There should be one -- and preferably only one -- obvious way to do it.
%\end{center}
%
%\subsection{The Zen of Python}
%\begin{center}
%    There should be one -- and preferably only one -- obvious way to do it.
%
%\addvspace{.5in}
%
%    Although that way may not be obvious at first unless you're Dutch. \\
%\vspace*{\fill}
%	\includegraphics[scale=0.5]{pics/sign.eps}
%\end{center}
%
%
%\subsection{The Zen of Python}
%\begin{center}
%    Now is better than never.
%\end{center}
%
%\subsection{The Zen of Python}
%\begin{center}
%    Now is better than never.  \\
%
%\addvspace{.5in}
%
%    Although never is often better than *right* now.
%\end{center}
%
%\subsection{The Zen of Python}
%\begin{center}
%    If the implementation is hard to explain, it's a bad idea.
%\end{center}
%
%\subsection{The Zen of Python}
%\begin{center}
%    If the implementation is easy to explain, \\
%	it {\em may} be a good idea.
%\end{center}
%
%\subsection{A simple interface, easy to explain.  Yet...}
%\\
%\vspace*{\fill}
%\begin{center}
%	\includegraphics[scale=1.2]{pics/elevator_buttons-reverse.eps}
%\end{center}
%\vspace*{\fill}
%
%
%\subsection{The Zen of Python}
%\begin{center}
%    Namespaces are one honking great idea -- let's do more of those!
%\end{center}
%\Normalsize
%
%\subsection{Documentation}
%\vspace*{\fill}
%\begin{center}
%	\includegraphics[scale=0.9]{pics/manual.eps}
%	\hspace{.5in}
%	\includegraphics[scale=0.9]{pics/manual2.eps}
%	\\
%	\vspace{.2in}
%	\Huge
%	{\bf WTFM}
%	\Normalsize
%\end{center}
%\vspace*{\fill}
%
%
%\subsection{Consistency}
%\vspace*{\fill}
%\begin{center}
%	\includegraphics[scale=1.1]{pics/consistency.eps}
%\end{center}
%\vspace*{\fill}
%
%\subsection{Robustness Principle or Postel's Law}
%\\
%\Huge
%\begin{center}
%	Be conservative in what you do; be liberal in what you accept from others.
%\end{center}
%\Normalsize
%
%\subsection{Avoid the Quick Fix}
%\\
%\Huge
%\begin{center}
%	There's nothing as permanent as a temporary solution.
%\end{center}
%\Normalsize
%
%\subsection{Take a good look in the mirror!}
%\\
%\vspace*{\fill}
%\begin{center}
%	\includegraphics[scale=0.65]{pics/donkey.eps} \\
%	\small
%	Nice Ass!
%\end{center}
%\vspace*{\fill}
%
%\subsection{Take a good look in the mirror!}
%\\
%\Huge
%\begin{center}
%	Until you can {\em prove} otherwise, \\
%	assume that {\em you} are the Ass!
%\end{center}
%\Normalsize
%
%\subsection{Avoid the Project That Was Never Finished}
%\\
%\Huge
%\begin{center}
%	Don't let the Perfect be the enemy of the Good.
%\end{center}
%\Normalsize
%
%\subsection{Avoid Feature Creep}
%\vspace*{\fill}
%\begin{center}
%	\includegraphics[scale=1.0]{pics/feeping.eps} \\
%	\small
%	\verb+http://www.feepingcreatures.com+
%\end{center}
%\vspace*{\fill}
%
%\subsection{Release Early, Release Often}
%\\
%\Huge
%\begin{center}
%	``More users find more bugs.'' \\
%	\addvspace{.2in}
%	\small F. Brooks, ``The Mythical Man Month''
%\end{center}
%\Normalsize
%
%\subsection{Increase the Bus Factor}
%\vspace*{\fill}
%\begin{center}
%	\includegraphics[scale=0.85]{pics/bert-ernie.eps} \\
%	\small
%	``Just friends.''
%\end{center}
%\vspace*{\fill}
%
%\subsection{Fix Broken Windows}
%\vspace*{\fill}
%\begin{center}
%	\includegraphics[scale=0.7]{pics/broken-windows.eps}
%\end{center}
%\vspace*{\fill}
%
%\subsection{Program Maintenance}
%\\
%\Huge
%\begin{center}
%	``... is an entropy-increasing process, and even its most skillful
%	execution only delays the subsidence of the system into unfixable
%	obsolescence.'' \\
%	\addvspace{.2in}
%	\small F. Brooks, ``The Mythical Man Month''
%\end{center}
%\Normalsize
%
%\subsection{Toss it!}
%\vspace*{\fill}
%\begin{center}
%	\includegraphics[scale=3]{pics/waste.eps}
%\end{center}
%\vspace*{\fill}
%
%\subsection{Starting fresh}
%\vspace*{\fill}
%\begin{center}
%	\includegraphics[scale=2]{pics/clean-slate.eps}
%\end{center}
%\vspace*{\fill}
%
%
\subsection{Reading}
Shell:
\begin{itemize}
	\item \verb+http://www.tldp.org/HOWTO/Bash-Prog-Intro-HOWTO.html+
	\item \verb+http://www.tldp.org/LDP/abs/html/+
	\item \verb+csh(1)+, \verb+ksh(1)+, \verb+sh(1)+
	\item {\em Frisch}: Chapter 3, 14
\end{itemize}
%Perl:
%\begin{itemize}
%	\item \verb+http://www.perl.com+
%	\item \verb+http://www.cpan.org+
%	\item \verb+perl(1)+, \verb+perldoc(1)+, \verb+perlfaq(1)+
%\end{itemize}
%Python:
%\begin{itemize}
%	\item \verb+http://www.python.org+
%	\item \verb+http://www.samag.com/documents/s=8964/sam0312a/0312a.htm+
%	\item pydoc
%\end{itemize}

\end{document}
