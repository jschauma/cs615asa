\documentclass[xga]{xdvislides}
\usepackage[landscape]{geometry}
\usepackage{array}
\usepackage{graphics}
\usepackage{graphicx}
\usepackage{colordvi}
\usepackage{tabularx}
\usepackage{multirow}
\usepackage{fancyvrb}

\fvset{fontfamily=courier,commandchars=\\\{\}}

\newcommand{\smallish}{\fontsize{16}{16}\selectfont}

\begin{document}
\setfontphv

%%% Headers and footers
\lhead{\slidetitle}				% default:\lhead{\slidetitle}
\chead{CS615 - Aspects of System Administration}% default:\chead{\relax}
\rhead{Slide \thepage}				% default:\rhead{\sectiontitle}
\lfoot{\Gray{Review}}% default:\lfoot{\slideauthor}
\cfoot{\relax}					% default:\cfoot{\relax}
\rfoot{\Gray{\today}}

\vspace*{\fill}
\begin{center}
	\Hugesize
		CS615 - Aspects of System Administration\\ [1em]
		The Whole Semester In One Class\\ [1em]
	\hspace*{5mm}\blueline\\ [1em]
	\Normalsize
		Department of Computer Science\\
		Stevens Institute of Technology\\
		Jan Schaumann\\
		\verb+jschauma@stevens.edu+ \\
		\verb+https://stevens.netmeister.org/615A/+
\end{center}
\vspace*{\fill}
%\setcounter{page}{0}
%\clearpage

\subsection{Basic Disk Concepts: Storage Models}
Direct Attached Storage (DAS)
\vfill
\begin{center}
	\includegraphics[scale=0.8]{pics/das.eps} \\
\end{center}
\verb+ssh lab 'df -hT /'+
\vfill

\subsection{Basic Disk Concepts: Storage Models}
Network Attached Storage (NAS)
\vfill
\begin{center}
	\includegraphics[scale=0.5]{pics/nas.eps} \\
\end{center}
\verb+ssh lab 'df -hT /home/$(whoami)'+
\vfill

\subsection{Basic Disk Concepts: Storage Models}
Storage Area Networks (SAN)
\vfill
\begin{center}
	\includegraphics[scale=0.5]{pics/san-nas-das.eps} \\
\end{center}
\vfill

\subsection{Basic Disk Concepts: Storage Models}
Cloud Storage (Examples: EBS, S3)
\vfill
\begin{center}
	\includegraphics[scale=0.6]{pics/cloud-storage.eps} \\
\end{center}
\vfill

\subsection{Basic Disk Concepts: Disk Devices}
\vfill
	\begin{center}
		\includegraphics[scale=0.9]{pics/6platter.eps} \\
	\end{center}
\vfill

\subsection{Storage Models and Disks}
Aboubacar Diawara \\
\vspace{1in}

\verb+https://www.dnsstuff.com/storage-array+

\newpage
\vspace*{\fill}
\begin{center}
    \Hugesize
        Lecture 03 \\ [1em]
    \hspace*{5mm}
    \blueline\\
    \hspace*{5mm}\\
	Filesystem Basics, Software Types
\end{center}
\vspace*{\fill}

\subsection{Basic Filesystem Concepts: The UNIX Filesystem}
\begin{center}
	\includegraphics[scale=0.8]{pics/ufs-details.eps} \\
\end{center}
\vspace*{\fill}

\subsection{Filesytem Basics}
Chirag Rana \\
\vspace{1in}

\begin{itemize}
	\item Filesystem Overview - \verb+https://is.gd/HfT7rJ+
	\item File System Inconsistency - \verb+https://is.gd/0usKSf+
	\item FSCK or Journaling - \verb+https://is.gd/jlFtOY+, \verb+https://is.gd/YXrXir+
\end{itemize}

\subsection{Types of Software}
\vfill
\begin{center}
	\includegraphics[scale=0.8]{pics/types-of-software.eps}
\end{center}
\vfill

\subsection{Software Installation and Package Managers}
David Sevilla\\
\vspace{1in}

\verb+https://is.gd/Og4J7w+ \\
\verb+https://lwn.net/Articles/712318/+

\newpage
\vspace*{\fill}
\begin{center}
    \Hugesize
        Lecture 04 \\ [1em]
    \hspace*{5mm}
    \blueline\\
    \hspace*{5mm}\\
	Software Installation, Multiuser Fundamentals
\end{center}
\vspace*{\fill}

\subsection{Software Installation}
Harshala Yadav\\
\vspace{1in}

\verb+https://is.gd/1Mbj2q+

\subsection{Multiuser Fundamentals}
Elliot Wasem\\
\vspace{1in}

\verb+https://is.gd/DpcmI+


\newpage
\vspace*{\fill}
\begin{center}
    \Hugesize
        Lecture 05 / Lecture 06 \\ [1em]
    \hspace*{5mm}
    \blueline\\
    \hspace*{5mm}\\
	Networking
\end{center}
\vspace*{\fill}

\subsection{Mommy, where do IP addresses come from?}
\vspace*{\fill}
\begin{center}
The Internet Assigned Numbers Authority (IANA) oversees global IP
address/AS number allocation, root zone management etc. \\
	\vspace{.5in}
	\includegraphics[scale=0.4]{pics/rirs.eps} \\
	\vspace{.5in}
	Regional Internet Registries (RIR) manage the allocation and
registration of Internet number resources within a region of the world.
\end{center}
\vspace*{\fill}

\subsection{WHOIS ASN?}
Autonomous System Numbers (ASNs) are assigned by IANA
to the RIRs, see e.g. {\tt
ftp://ftp.arin.net/pub/stats/arin/}
\\

You can query databases on the internet about e.g. IP
block or ASN information via the {\tt WHOIS} protocol:

\begin{verbatim}
$ whois 155.246.89.100 | more
NetRange:       155.246.0.0 - 155.246.255.255
CIDR:           155.246.0.0/16
NetName:        STEVENS
NetHandle:      NET-155-246-0-0-1
Parent:         NET155 (NET-155-0-0-0-0)
NetType:        Direct Assignment
Organization:   Stevens Institute of Technology (SIT)
RegDate:        1991-12-31
Updated:        2007-01-29
Ref:            https://whois.arin.net/rest/net/NET-155-246-0-0-1
\end{verbatim}

\subsection{WHOIS ASN?}
Carriers connect their Autonomous Systems at {\em
Internet Exchange Points} (IXPs) to route traffic
between the different networks.\\

This {\em peering} happens amongst carriers on a
tiered basis. \\

Examples:
\begin{verbatim}
https://peeringdb.com/net?asn=6939
https://peeringdb.com/net/27
https://peeringdb.com/net/433
https://peeringdb.com/net/457
\end{verbatim}

\subsection{Networking}
Stringing cables across the oceans' floors since 1866!
\vspace*{\fill}
\begin{center}
	\includegraphics[scale=1.0]{pics/internet-undersea-cable.eps} \\
	\verb+http://www.submarinecablemap.com/+ \\
	\verb+http://is.gd/CjanOu+
\end{center}
\vspace*{\fill}

\subsection{Networking I}
Jared Bass\\
\vspace{1in}

\verb+https://is.gd/bP8dZU+ \\
\verb+https://is.gd/EA2Ddy+

\subsection{A simple example}
\\
\Hugesize
\begin{center}
\begin{verbatim}
$ strace -f telnet www.google.com 80 2>strace.out
Trying 173.194.73.99...
Connected to www.google.com.
Escape character is '^]'.
GET / HTTP/1.0

[...]
\end{verbatim}
\end{center}
\Normalsize
\vspace*{\fill}

\subsection{TCP/IP Basics: Putting it all together}
\vspace*{\fill}
\begin{center}
	\includegraphics[scale=0.6]{pics/tcpip-stack.eps}
\end{center}
\vspace*{\fill}

\subsection{Networking II}
Jiahan Liu\\
\vspace{1in}

\verb+https://skerritt.blog/how-does-tor-really-work/+

\newpage
\vspace*{\fill}
\begin{center}
    \Hugesize
        Lecture 07 \\ [1em]
    \hspace*{5mm}
    \blueline\\
    \hspace*{5mm}\\
	DNS; HTTP
\end{center}
\vspace*{\fill}

\subsection{Hostname resolution}
\vspace*{\fill}
\begin{center}
	\includegraphics[scale=0.9]{pics/resolution.eps}
\end{center}
\vspace*{\fill}

\subsection{DNS}
Michael Appiah\\
\vspace{1in}

\verb+https://securitytrails.com/blog/8-tips-to-prevent-dns-attacks+ \\
\verb+https://www.esecurityplanet.com/network-security/how-to-prevent-dns-attacks.html+

\subsection{The Hypertext Transfer Protocol}
HTTP is a request/response protocol:
\begin{enumerate}
	\item client sends a request to the server
		\begin{itemize}
			\item request method
			\item URI
			\item protocol version
			\item request modifiers
			\item client information
		\end{itemize}
	\item server responds
		\begin{itemize}
			\item status line (including success or error code)
			\item server information
			\item entity metainformation
			\item content
		\end{itemize}
\end{enumerate}

\subsection{HTTP Proxy Servers}
\begin{itemize}
	\item HTTP traffic usually is very asymmetric
	\item a lot of the content is static
	\item network ACLs may restrict traffic flow
\end{itemize}
\vspace{.25in}
\begin{center}
	\includegraphics[scale=0.6]{pics/revproxy.eps}
\end{center}

\newpage
\vspace*{\fill}
\begin{center}
    \Hugesize
        Lecture 08 \\ [1em]
    \hspace*{5mm}
    \blueline\\
    \hspace*{5mm}\\
	SMTP / HTTPS
\end{center}
\vspace*{\fill}

\subsection{TLS}
Transport Layer Security
\begin{itemize}
	\item set of cryptographic protocols
	\item operates on layer 6 of OSI stack (Presentation Layer)
	\item independent of HTTP
	\item RFC5246 (TLS 1.2)
\end{itemize}
\addvspace{.5in}
Two distinct security mechanisms:
\begin{enumerate}
	\item encryption of data in transit
	\item authentication of parties
\end{enumerate}

\subsection{HTTP/HTTPS}
Mark Freeman\\
\vspace{1in}

\verb+https://web.stanford.edu/class/cs253/+ \\
\verb+https://www.cloudflare.com/learning/ssl/how-does-public-key-encryption-work/+


\subsection{SMTP: Sending...}
\begin{verbatim}
# tcpdump -i xennet0 -w /tmp/t.out port not 22 2>/dev/null &
# mail -s "CS615 - SMTP Exercise" jschauma@stevens.edu -f jschauma@stevens.edu
Hello,

SMTP is simple.

-Jan
.
EOT
# fg
tcpdump -i xennet0 -w /tmp/t.out port not 22 2>/dev/null
^C
\end{verbatim}

\subsection{Sending...}
\begin{Verbatim}
$ telnet 155.246.14.37 25
Trying 155.246.14.37...
Connected to spamfilter01.stevens.edu.
Escape character is '^]'.
\textbf{220 spamfilter01.stevens.edu ESMTP (fe32969a29a5f461e53bf93b18c8fdb5)}
EHLO ip-10-235-167-232.ec2.internal
\textbf{250-spamfilter01.stevens.edu Hello ec2-54-205-68-41.compute-1.amazonaws.com [54.205.68.41],}
\textbf{        pleased to meet you}
\textbf{250-SIZE 50000000}
\textbf{250-PIPELINING}
\textbf{250-8BITMIME}
\textbf{250 HELP}
MAIL FROM:<jschauma@stevens.edu> SIZE=380
\textbf{250 Sender <jschauma@stevens.edu> OK}
RCPT TO:<jschauma@stevens.edu>
\textbf{250 Recipient <jschauma@stevens.edu> OK}
\end{Verbatim}

\subsection{Sending...}
\begin{Verbatim}
DATA
\textbf{354 Start mail input; end with <CRLF>.<CRLF>}
Received: by ip-10-235-167-232.ec2.internal (Postfix, from userid 0)
        id 2A17275438; Mon,  4 Apr 2016 15:42:33 +0000 (UTC)
To: jschauma@stevens.edu
Subject: CS615 - SMTP Exercise
Message-Id: <20160404154233.2A17275438@ip-10-235-167-232.ec2.internal>
Date: Mon,  4 Apr 2016 15:42:33 +0000 (UTC)
From: jschauma@stevens.edu (Charlie Root)

Hello,

SMTP is simple.

-Jan
.
\textbf{250 Ok: queued as 6A9C76F4004}
\end{Verbatim}

\subsection{SMTP}
Nan Cui\\
\vspace{1in}

\verb+https://is.gd/LJAYTT+


\newpage
\vspace*{\fill}
\begin{center}
    \Hugesize
        Lecture 09 \\ [1em]
    \hspace*{5mm}
    \blueline\\
    \hspace*{5mm}\\
	Writing System Tools
\end{center}
\vspace*{\fill}

\subsection{Tools}
\vspace*{\fill}
\begin{center}
	\includegraphics[scale=1.4]{pics/tools.eps}
\end{center}
\vspace*{\fill}

\subsection{Unix Philosophy}
\\
\Huge
\begin{center}
	Write programs that do one thing and do it well.\\
	\vspace{.5in}
	Write programs to work together. \\
	\vspace{.5in}
	Write programs to handle text streams, because that is a universal interface.
\end{center}
\Normalsize

\subsection{Writing System Tools}
Ritvik Tiwari\\
\vspace{1in}

\verb+https://www.linuxjournal.com/content/sysadmin-101-automation+

\newpage
\vspace*{\fill}
\begin{center}
    \Hugesize
        Lecture 10 \\ [1em]
    \hspace*{5mm}
    \blueline\\
    \hspace*{5mm}\\
	Backup and Disaster Recovery / Monitoring
\end{center}
\vspace*{\fill}

\subsection{Backup and Disaster Recovery}
\vspace*{\fill}
\begin{center}
	\includegraphics[scale=0.8]{pics/schrodinger.eps}
\end{center}
\vspace*{\fill}

\subsection{Backup and Disaster Recovery}
Sai Gudivada\\
\vspace{1in}

\verb+https://is.gd/ih10np+\\
\verb+https://is.gd/dBFPdn+

\subsection{Events and Metrics}
\begin{center}
	\includegraphics[scale=0.75]{pics/events-metrics.eps}
\end{center}

\subsection{Events and Metrics}
Timothy Steinberg\\
\vspace{1in}

\verb+https://is.gd/B2MwMU+ \\
\verb+https://is.gd/qDTBKG+

\newpage
\vspace*{\fill}
\begin{center}
    \Hugesize
        Lecture 11 \\ [1em]
    \hspace*{5mm}
    \blueline\\
    \hspace*{5mm}\\
	Configuration Management
\end{center}
\vspace*{\fill}

\subsection{CM States}
\vspace*{\fill}
\begin{center}
	\includegraphics[scale=0.7]{pics/host-states.eps} \\
\end{center}
\vspace*{\fill}

\subsection{Configuration Management}
Zhipeng Zhang\\
\vspace{1in}

\verb+https://is.gd/aZRvit+ \\
\verb+https://www.upguard.com/articles/ansible-vs-chef+

\subsection{This Whole Semester in One Slide}
\vspace*{\fill}
\Huge
\begin{center}
System Administration is a unique, constantly
developing profession. \\
\vspace{.5in}

It can be fun, satisfying, interesting, and impactful,
but it's not easy. \\
\vspace{.5in}

Don't be lazy.
\end{center}
\Normalsize
\vspace*{\fill}

\end{document}
