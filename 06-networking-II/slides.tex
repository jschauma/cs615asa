\special{! TeXDict begin /landplus90{true}store end }

\documentclass[xga]{xdvislides}
\usepackage[landscape]{geometry}
\usepackage{graphics}
\usepackage{graphicx}
\usepackage{colordvi}
\usepackage{multirow}

\usepackage{fancyvrb}
\fvset{commandchars=\\\{\}}

\usepackage[usenames]{color}
\definecolor{gray}{RGB}{180,180,180}

\newcommand{\smallish}{\fontsize{16}{16}\selectfont}

\begin{document}
\setfontphv

%%% Headers and footers
\lhead{\slidetitle}                               % default:\lhead{\slidetitle}
\chead{CS615 - Aspects of System Administration}% default:\chead{\relax}
\rhead{Slide \thepage}                       % default:\rhead{\sectiontitle}
\lfoot{\Gray{Networking II}}% default:\lfoot{\slideauthor}
\cfoot{\relax}                               % default:\cfoot{\relax}
\rfoot{\Gray{\today}}

\vspace*{\fill}
\begin{center}
	\Hugesize
		CS615 - Aspects of System Administration\\ [1em]
		Networking II\\ [1em]
	\hspace*{5mm}\blueline\\ [1em]
	\Normalsize
		Department of Computer Science\\
		Stevens Institute of Technology\\
		Jan Schaumann\\
		\verb+jschauma@stevens.edu+\\
		\verb+http://stevens.netmeister.org/615/+
\end{center}
\vspace*{\fill}

\subsection{Get your instruments and play along!}

\Hugesize
\vspace*{\fill}
Start a FreeBSD instance: {\tt ami-03b0f822e17669866}
\vspace*{\fill}
\Normalsize

\subsection{HW1 Comments}

\begin{itemize}
	\item provide more details in your README
	\item learn to format your README in a
		readable manner
	\item if you run into problems, ask on the mailing list
	\item do not print any output that was not requested
	\item avoid useless comments
	\item make no assumptions about the directory
		from which your program will be invoked
\end{itemize}

\subsection{HW1 Comments}
\begin{itemize}
	\item do not hardcode information that is specific to your own account
	\item do not hardcode a single AMI; you need
		to be able to support multiple availability zones
	\item terminate the instance after your
		program completes (on success {\em or} error)
	\item do not specify {\tt -i} to {\tt ssh(1)}
	\item use {\tt aws ec2 wait} instead of loop-waiting yourself
\end{itemize}

\subsection{HW1 Comments: avoid waterfall code}
\begin{center}
	\includegraphics[scale=0.38]{pics/waterfall.eps} \\
\end{center}

\subsection{HW1 Comments}
Writing shell scripts is not very different from
writing programs in other languages.

\begin{itemize}
	\item use \verb+getopt(1)+ to parse command-line arguments
	\item use functions
	\item check commands and functions for return values 
\end{itemize}
\vspace{.5in}

Some useful links:
\begin{itemize}
	\item \verb+https://www.netmeister.org/blog/writing-shell-scripts.html+
	\item \verb+https://github.com/koalaman/shellcheck+
	\item \verb+https://google.github.io/styleguide/shell.xml+
\end{itemize}


\subsection{Layer 1: Repeater Hub}
\vspace*{\fill}
\begin{center}
	\includegraphics[scale=0.8]{pics/hub.eps} \\
	Half-duplex, cheap, obsolete.
\end{center}
\vspace*{\fill}

\subsection{Layer 2: Network Switch}
\vspace*{\fill}
\begin{center}
	\includegraphics[scale=0.9]{pics/switch.eps} \\
	MAC bridge, full-duplex, segmentation, CIDR, STP \\
	\verb+https://is.gd/HPhuRG+
\end{center}
\vspace*{\fill}

\subsection{Layer 3: Router}
\vspace*{\fill}
\begin{center}
	\includegraphics[scale=0.2]{pics/router.eps} \\
	connect networks, forward packets, routing tables, BGP
\end{center}
\vspace*{\fill}

\subsection{A simple example}
\Hugesize
\begin{center}
\begin{verbatim}
$ telnet www.yahoo.com 80

\end{verbatim}
\end{center}
\Normalsize
\vspace*{\fill}

\subsection{A simple example}
\Hugesize
\begin{center}
\begin{verbatim}
$ telnet www.yahoo.com 80
Trying 98.138.219.232...
Connected to atsv2-fp-shed.wg1.b.yahoo.com.
Escape character is '^]'.
HEAD / HTTP/1.0

\end{verbatim}
\end{center}
\Normalsize
\vspace*{\fill}

\subsection{A simple example}
\Hugesize
\begin{center}
\begin{verbatim}
$ telnet www.yahoo.com 80
Trying 98.138.219.232...
Connected to atsv2-fp-shed.wg1.b.yahoo.com.
Escape character is '^]'.
HEAD / HTTP/1.0

HTTP/1.0 200 OK
Date: Mon, 04 Mar 2019 17:41:59 GMT
Via: http/1.1 media-router-fp1010.prod.media.ne1.yahoo.com
Server: ATS
[...]
\end{verbatim}
\end{center}
\Normalsize
\vspace*{\fill}

\subsection{A simple example}
What exactly happens?

\subsection{A simple example}
What exactly happens?
\\
\begin{itemize}
	\item local host connects to remote host
	\item sends command
	\item receives data
\end{itemize}

\subsection{A simple example}
How exactly do we connect to the remote host?
\\
\begin{itemize}
	\item look up hostname
	\item open connection to IP address
\end{itemize}

\subsection{A simple example}
How exactly do we look up a hostname?

\subsection{A simple example}
\\
\Hugesize
\begin{center}
\begin{verbatim}
$ ktrace -i telnet www.yahoo.com 80
Trying 72.30.35.9...
Connected to atsv2-fp-shed.wg1.b.yahoo.com.
Escape character is '^]'.
HEAD / HTTP/1.0

[...]
$ kdump >trace
\end{verbatim}
\end{center}
\Normalsize
\vspace*{\fill}

\subsection{...open a few files...}
\smallish
\begin{verbatim}
[...]
   735 ktrace   RET   execve -1 errno 2 No such file or directory
   735 ktrace   CALL  execve(0xbfbfe7e0,0xbfbfed00,0xbfbfed10)
   735 ktrace   NAMI  "/usr/bin/telnet"
   735 ktrace   NAMI  "/libexec/ld-elf.so.1"
   735 telnet   RET   execve JUSTRETURN
[...]
   735 telnet   CALL  open(0x80066edc5,0x100000<O_RDONLY|O_CLOEXEC>)
   735 telnet   NAMI  "/etc/nsswitch.conf"
   735 telnet   RET   open 3
[...]
   735 telnet   CALL  open(0x800671afd,0x100000<O_RDONLY|O_CLOEXEC>)
   735 telnet   NAMI  "/etc/hosts"
   735 telnet   RET   open 3
[...]
   735 telnet   CALL  open(0x80066e6b5,0x100000<O_RDONLY|O_CLOEXEC>)
   735 telnet   NAMI  "/etc/resolv.conf"
   735 telnet   RET   open 3
[...]
   735 telnet   CALL  read(0x3,0x800c3be40,0x8000)
   735 telnet   GIO   fd 3 read 70 bytes
       "# Generated by resolvconf
        search ec2.internal
        nameserver 172.16.0.23
\end{verbatim}
\Normalsize

\subsection{... query a DNS server ...}
\smallish
\begin{verbatim}
[...]
   735 telnet   CALL  socket(PF_INET,0x10000002<SOCK_DGRAM|SOCK_CLOEXEC>,IPPROTO_IP)
   735 telnet   RET   socket 3
   735 telnet   CALL  connect(0x3,0x800a43914,0x10)
   735 telnet   STRU  struct sockaddr { AF_INET, 172.16.0.23:53 }
   735 telnet   RET   connect 0
   735 telnet   CALL  sendto(0x3,0x800c96400,0x1f,0,0,0)
   735 telnet   GIO   fd 3 wrote 31 bytes
       0x0000 e614 0100 0001 0000 0000 0000 0377 e777       |.............www|
       0x0010 0579 6168 6f6f 0363 6f6d 0000 0100 01         |.yahoo.com.....|
[...]
   735 telnet   CALL  recvfrom(0x3,0x800c71e00,0x10000,0,0x7fffffffd640,0x7fffffffd228)
   735 telnet   GIO   fd 3 read 97 bytes
       0x0000 e614 8180 0001 0003 0000 0000 0377 7777       |.............www|
       0x0010 0579 6168 6f6f 0363 6f6d 0000 0100 01c0       |.yahoo.com......|
       0x0020 0c00 0500 0100 0000 3c00 160d 6174 7376       |........<...atsv|
       0x0030 322d 6670 2d73 6865 6403 7767 3101 62c0       |2-fp-shed.wg1.b.|
       0x0040 10c0 2b00 0100 0100 0000 3c00 0448 1e23       |..+.......<..H.#|
       0x0050 09c0 2b00 0100 0100 0000 3c00 0448 1e23       |..+.......<..H.#|
       0x0060 0a                                            |.|
[...]
\end{verbatim}
\Normalsize

\subsection{A simple example}
How exactly do we look up a hostname?
\\
\begin{itemize}
	\item look up various local files
	\item open a connection to a DNS server's IP
	\item ask DNS server to resolve hostname
	\item get back IP
\end{itemize}
\vspace{.5in}

And then?

\subsection{...communicate with the remote host...}
\smallish
\begin{verbatim}
   735 telnet   GIO   fd 1 wrote 21 bytes
       "Trying 72.30.35.9...
       "
   735 telnet   RET   write 21/0x15
   735 telnet   CALL  socket(PF_INET,0x1<SOCK_STREAM>,IPPROTO_TCP)
   735 telnet   CALL  connect(0x3,0x8002650f0,0x10)
   735 telnet   STRU  struct sockaddr { AF_INET, 72.30.35.9:80 }
[...]
   918 telnet   GIO   fd 0 read 16 bytes
       "HEAD / HTTP/1.0
       "
   918 telnet   RET   read 16/0x10
   918 telnet   CALL  select(0x4,0x80025e1d8,0x80025e1c8,0x80025e1d0,0x229058)
   918 telnet   RET   select 1
   918 telnet   CALL  sendto(0x3,0x226490,0x11,0,0,0))
   918 telnet   GIO   fd 3 wrote 17 bytes
       "HEAD / HTTP/1.0\r
       "
[...]
   918 telnet   RET   select 1
   918 telnet   CALL  recvfrom(0x3,0x226040,0x400,0,0,0)
   918 telnet   GIO   fd 3 read 324 bytes
       "HTTP/1.0 200 OK\r
        Date: Mon, 04 Mar 2019 17:44:09 GMT\r
\end{verbatim}
\Normalsize

\subsection{Ok, so how does this work?}
\begin{itemize}
	\item determine which nameserver to query
	\item ask who has a route to the nameserver
	\item open socket to well defined port on remote IP
	\item send queries
	\item open socket to requested port on remote IP
\end{itemize}

\subsection{Let's collect some data...}
\begin{verbatim}
laptop$ ssh ec2-user@<instance-name>
$ su
# script commands.out
# ifconfig -a; route -n get default
# cat /etc/resolv.conf
# tcpdump -w tcpdump.out port not 22 >&/dev/null &
# arp -d -a
# ping -n -c 3 8.8.8.8
# ktrace -i telnet www.yahoo.com 80
HEAD / HTTP/1.0
# kill %1
# kdump > kdump.out
# chmod a+r kdump.out
$ exit
laptop$ scp ec2-user@<instance-name>:*out /tmp/
\end{verbatim}

\subsection{A simple example}
Finding the next hop:
\begin{verbatim}
$ tcpdump -t -n -r /tmp/tcpdump.out arp
reading from file /tmp/tcpdump.out, link-type EN10MB (Ethernet)
ARP, Request who-has 10.183.114.1 tell 10.183.114.37, length 28
ARP, Reply 10.183.114.1 is-at fe:ff:ff:ff:ff:ff, length 28
ARP, Request who-has 10.183.114.37 tell 10.183.114.1, length 28
ARP, Reply 10.183.114.37 is-at 22:00:0a:b7:72:25, length 28
\end{verbatim}
\vspace{.5in}
Enter ARP Spoofing... (Red Team)

\subsection{A simple example}
Performing the DNS query:
\begin{verbatim}
$ tcpdump -t -n -r tcpdump.out udp port 53
reading from file tcpdump.out, link-type EN10MB (Ethernet)
IP 10.183.114.37.53383 > 172.16.0.23.53: 20378+ A? www.yahoo.com. (31)
IP 172.16.0.23.53 > 10.183.114.37.53383: 20378 5/0/0
	CNAME atsv2-fp-shed.wg1.b.yahoo.com., A 72.30.35.9, A 72.30.35.10,
	A 98.138.219.231, A 98.138.219.232 (129)
IP 10.183.114.37.10551 > 172.16.0.23.53: 60792+ AAAA? www.yahoo.com. (31)
IP 172.16.0.23.53 > 10.183.114.37.10551: 60792 5/0/0 CNAME atsv2-fp-shed.wg1.b.yahoo.com.,
	AAAA 2001:4998:58:1836::11, AAAA 2001:4998:44:41d::3,
	AAAA 2001:4998:44:41d::4, AAAA 2001:4998:58:1836::10 (177)
\end{verbatim}

\subsection{A simple example}
Establishing the connection to the server:
\begin{verbatim}
$ tcpdump -t -n -r tcpdump.out tcp port 80
IP 10.183.114.37.45403 > 72.30.35.9.80: Flags [S], seq 5028151, win 65535,
	options [mss 1460,nop,wscale 6,sackOK,TS val 598758437 ecr 0], length 0
IP 72.30.35.9.80 > 10.183.114.37.45403: Flags [S.], seq 1983855029, ack 5028152, win 14480,
	options [mss 1460,sackOK,TS val 1888595968 ecr 598758437,nop,wscale 8], length 0
IP 10.183.114.37.45403 > 72.30.35.9.80: Flags [.], ack 1, win 1026,
	options [nop,nop,TS val 598758465 ecr 1888595968], length 0
\end{verbatim}

\subsection{A simple example}
Sending the HTTP request:
\begin{verbatim}
IP 10.183.114.37.45403 > 72.30.35.9.80: Flags [P.], seq 1:18, ack 1, win 1026,
	options [nop,nop,TS val 598762755 ecr 1888595968], length 17: HTTP: HEAD / HTTP/1.0
IP 72.30.35.9.80 > 10.183.114.37.45403: Flags [.], ack 18, win 57,
	options [nop,nop,TS val 1888600285 ecr 598762755], length 0
IP 10.183.114.37.45403 > 72.30.35.9.80: Flags [P.], seq 18:20, ack 1, win 1026,
	options [nop,nop,TS val 598764209 ecr 1888600285], length 2: HTTP
IP 72.30.35.9.80 > 10.183.114.37.45403: Flags [.], ack 20, win 57,
	options [nop,nop,TS val 1888601739 ecr 598764209], length 0
\end{verbatim}

\subsection{A simple example}
Receiving the HTTP response:
\begin{verbatim}
IP 72.30.35.9.80 > 10.183.114.37.45403: Flags [P.], seq 1:325, ack 20, win 57,
	options [nop,nop,TS val 1888601741 ecr 598764209], length 324: HTTP: HTTP/1.0 200 OK
\end{verbatim}

\subsection{A simple example}
Terminating the connection:
\begin{verbatim}
IP 72.30.35.9.80 > 10.183.114.37.45403: Flags [F.], seq 325, ack 20, win 57,
	options [nop,nop,TS val 1888601741 ecr 598764209], length 0
IP 10.183.114.37.45403 > 72.30.35.9.80: Flags [.], ack 326, win 1021,
	options [nop,nop,TS val 598764236 ecr 1888601741], length 0
IP 10.183.114.37.45403 > 72.30.35.9.80: Flags [F.], seq 20, ack 326, win 1026,
	options [nop,nop,TS val 598764236 ecr 1888601741], length 0
IP 72.30.35.9.80 > 10.183.114.37.45403: Flags [.], ack 21, win 57,
	options [nop,nop,TS val 1888601768 ecr 598764236], length 0
\end{verbatim}

\subsection{Notables from this simple example}
``Simple'' is, as usual, relative.

\subsection{Notables from this simple example}
``Simple'' is, as usual, relative.
\\

\begin{itemize}
	\item host configuration assumed
	\item network architecture (internal or across the internet) not
			relevant (here)
	\item even simple examples cross multiple layers and protocols
			(HTTP, DNS; TCP, UDP, ARP)
	\item we haven't even scratched the surface
\end{itemize}

\subsection{TCP/IP Basics: Protocol Layers}
\begin{center}
	\begin{tabular}{|cl|l|}
	\hline
	& {\bf Layer} & {\bf Function} \\
	\hline
	4. & Application Layer & End-User application programs \\
	3. & Transport Layer & Delivery of data to applications \\
	2. & Network Layer & Basic communication, addressing, and routing \\
	\multirow{2}{*}{1.} & Link Layer & Network Hardware and device drivers \\
	& Physical Layer & Cable or physical medium \\
	\hline
	\end{tabular}
\end{center}
\addvspace{.5in}
Examples of protocols for each layer:
\begin{itemize}
	\item Simple Mail Transfer Protocol (RFC 821) \\
		Hypertext Transfer Protocol (RFC 2616)
	\item Transmission Control Protocol (RFC 793, tcp(4)) \\
		User Datagram Protocol (RFC 768; udp(4))
	\item Internet Protocol (RFC 791; ip(4)) \\
		Internet Control Message Protocol (RFC 792; icmp(4))
	\item Address Resolution Protocol (RFC 826; arp(4))
\end{itemize}

\subsection{TCP/IP Basics: Protocol Layers (OSI Model)}
\vspace*{\fill}
\begin{center}
	\includegraphics[scale=0.7]{pics/osi.eps}
\end{center}
\vspace*{\fill}

\subsection{TCP/IP Basics: ARP}
\begin{center}
Ethernet Address Resolution Protocol \\
-- or -- \\
Converting Network Protocol Addresses to 48-bit Ethernet Address for Transmission on Ethernet Hardware
\end{center}

\begin{verbatim}
$ arp -a
falcon.srcit.stevens-tech.edu (155.246.89.89) at 00:07:e9:09:ca:10 [ether] on eth0
grohl.srcit.stevens-tech.edu (155.246.89.9) at 00:16:3e:cf:6b:5b [ether] on eth0
hoth.srcit.stevens-tech.edu (155.246.89.10) at 00:1e:68:8e:79:d8 [ether] on eth0
cinema.srcit.stevens-tech.edu (155.246.89.67) at 00:25:90:1e:05:51 [ether] on eth0
vlan16.cc.stevens-tech.edu (155.246.89.1) at 00:00:5e:00:01:02 [ether] on eth0
vader.srcit.stevens-tech.edu (155.246.89.5) at 00:23:8b:a9:dd:60 [ether] on eth0
nirvana.phy.stevens-tech.edu (155.246.89.33) at 00:1e:68:0f:99:a2 [ether] on eth0
\end{verbatim}

\subsection{TCP/IP Basics: ARP}
\vspace*{\fill}
\begin{center}
	\includegraphics[scale=0.8]{pics/3computers-arp.eps}
\end{center}
\vspace*{\fill}


\subsection{TCP/IP Basics: ARP}
\begin{center}
Ethernet Address Resolution Protocol \\
-- or -- \\
Converting Network Protocol Addresses to 48-bit Ethernet Address for Transmission on Ethernet Hardware
\end{center}
\vspace{.2in}

\begin{verbatim}
ARP, Request who-has 10.114.62.1 tell 10.114.63.209, length 28
ARP, Reply 10.114.62.1 is-at fe:ff:ff:ff:ff:ff, length 28
ARP, Request who-has 10.114.63.209 (ff:ff:ff:ff:ff:ff) tell 0.0.0.0, length 28
ARP, Reply 10.114.63.209 is-at 12:31:3d:04:30:23, length 28
ARP, Request who-has 10.114.63.209 (ff:ff:ff:ff:ff:ff) tell 0.0.0.0, length 28
ARP, Reply 10.114.63.209 is-at 12:31:3d:04:30:23, length 28
ARP, Request who-has 10.114.63.209 (ff:ff:ff:ff:ff:ff) tell 0.0.0.0, length 28
ARP, Reply 10.114.63.209 is-at 12:31:3d:04:30:23, length 28
\end{verbatim}

\subsection{TCP/IP Basics: ND}
\begin{center}
Neighbor Discovery Protocol
\end{center}
\vspace{.2in}

\begin{verbatim}
$ ndp -n -a
Neighbor                            Linklayer Address  Netif Expire      S Flags
2001:470:30:84:e276:63ff:fe72:3900  e0:76:63:72:39:00  xennet0 permanent R
fe80::21b:21ff:fe45:bf54%xennet0    00:1b:21:45:bf:54  xennet0 21m52s    S R
fe80::21b:21ff:fe7a:7269%xennet0    00:1b:21:7a:72:69  xennet0 23h59m59s S R
fe80::e276:63ff:fe72:3900%xennet0   e0:76:63:72:39:00  xennet0 permanent R
fe80::1%lo0                          (incomplete)      lo0     permanent R
$
\end{verbatim}

\subsection{TCP/IP Basics: ND}
\begin{center}
Neighbor Discovery Protocol
\end{center}
\vspace{.2in}
\begin{verbatim}
IP6 fe80::21b:21ff:fe7a:7269 > ff02::1:ff62:3400: ICMP6,
        neighbor solicitation, who has 2001:470:30:84:e276:63ff:fe62:3400, length 32
IP6 2001:470:30:84:e276:63ff:fe72:3900 > ff02::1:ff7a:7269: ICMP6,
        neighbor solicitation, who has fe80::21b:21ff:fe7a:7269, length 32
IP6 fe80::21b:21ff:fe7a:7269 > 2001:470:30:84:e276:63ff:fe72:3900:
        ICMP6, neighbor advertisement, tgt is fe80::21b:21ff:fe7a:7269, length 32
\end{verbatim}

\subsection{TCP/IP Basics: ICMP}
\begin{center}
Internet Control Message Protocol
\end{center}
\vspace{.2in}

\begin{verbatim}
$ ping -c 3 www.yahoo.com
PING any-fp.wa1.b.yahoo.com (67.195.160.76): 56 data bytes
64 bytes from 67.195.160.76: icmp_seq=0 ttl=53 time=30.888 ms
64 bytes from 67.195.160.76: icmp_seq=1 ttl=53 time=23.193 ms
64 bytes from 67.195.160.76: icmp_seq=2 ttl=53 time=25.433 ms

----any-fp.wa1.b.yahoo.com PING Statistics----
3 packets transmitted, 3 packets received, 0.0% packet loss
round-trip min/avg/max/stddev = 23.193/26.505/30.888/3.958 ms
$
\end{verbatim}
\vspace{.25in}
Enter ICMP attacks... (Blue Team)

\subsection{TCP/IP Basics: ICMP: Ping}
\vspace*{\fill}
\begin{center}
	\includegraphics[scale=0.8]{pics/3computers-ping.eps}
\end{center}
\vspace*{\fill}


\subsection{TCP/IP Basics: ICMP}
\begin{center}
Internet Control Message Protocol
\end{center}
\vspace{.2in}

\begin{verbatim}
$ tcpdump -r tcpdump.out -n icmp
IP 10.234.84.220 > 207.237.69.79: ICMP echo request
IP 207.237.69.79 > 10.234.84.220: ICMP echo reply
IP 10.234.84.220 > 207.237.69.79: ICMP echo request
IP 207.237.69.79 > 10.234.84.220: ICMP echo reply
IP 10.234.84.220 > 207.237.69.79: ICMP echo request
IP 207.237.69.79 > 10.234.84.220: ICMP echo reply
\end{verbatim}


\subsection{TCP/IP Basics: ICMP6}
\begin{center}
Internet Control Message Protocol for IPv6
\end{center}
\vspace{.2in}

\begin{verbatim}
$ ping6 -c 3 www.netbsd.org
PING6(56=40+8+8 bytes) 2001:470:30:84:204:d7b0:0:1 -->
                       2001:4f8:3:7:2e0:81ff:fe52:9a6b
16 bytes from 2001:4f8:3:7:2e0:81ff:fe52:9a6b, icmp_seq=0 hlim=57 time=74.316 ms
16 bytes from 2001:4f8:3:7:2e0:81ff:fe52:9a6b, icmp_seq=1 hlim=57 time=71.260 ms
16 bytes from 2001:4f8:3:7:2e0:81ff:fe52:9a6b, icmp_seq=2 hlim=57 time=71.321 ms

--- www.netbsd.org ping6 statistics ---
3 packets transmitted, 3 packets received, 0.0% packet loss
round-trip min/avg/max/std-dev = 71.260/72.299/74.316/1.747 ms
\end{verbatim}

\subsection{TCP/IP Basics: ICMP6}
\begin{center}
Internet Control Message Protocol for IPv6
\end{center}
\vspace{.2in}

\begin{verbatim}
IP6 2001:470:30:84:204:d7b0:0:1 >
   2001:4f8:3:7:2e0:81ff:fe52:9a6b: ICMP6, echo reque st, seq 0, length 16
IP6 2001:4f8:3:7:2e0:81ff:fe52:9a6b >
   2001:470:30:84:204:d7b0:0:1: ICMP6, echo reply , seq 0, length 16
IP6 2001:470:30:84:204:d7b0:0:1 >
   2001:4f8:3:7:2e0:81ff:fe52:9a6b: ICMP6, echo request, seq 1, length 16
IP6 2001:4f8:3:7:2e0:81ff:fe52:9a6b >
   2001:470:30:84:204:d7b0:0:1: ICMP6, echo reply , seq 1, length 16
IP6 2001:470:30:84:204:d7b0:0:1 >
   2001:4f8:3:7:2e0:81ff:fe52:9a6b: ICMP6, echo reque st, seq 2, length 16
IP6 2001:4f8:3:7:2e0:81ff:fe52:9a6b >
   2001:470:30:84:204:d7b0:0:1: ICMP6, echo reply , seq 2, length 16
\end{verbatim}

\subsection{TCP/IP Basics: ICMP: Traceroute}
\vspace*{\fill}
\begin{center}
	\includegraphics[scale=0.8]{pics/traceroute1.eps}
\end{center}
\vspace*{\fill}

\subsection{TCP/IP Basics: ICMP: Traceroute}
\vspace*{\fill}
\begin{center}
	\includegraphics[scale=0.8]{pics/traceroute2.eps}
\end{center}
\vspace*{\fill}

\subsection{TCP/IP Basics: ICMP: Traceroute}
\vspace*{\fill}
\begin{center}
	\includegraphics[scale=0.8]{pics/traceroute3.eps}
\end{center}
\vspace*{\fill}

\subsection{TCP/IP Basics: ICMP: Traceroute}
\vspace*{\fill}
\begin{center}
	\includegraphics[scale=0.8]{pics/traceroute4.eps}
\end{center}
\vspace*{\fill}



\subsection{TCP/IP Basics: ICMP}
\begin{center}
Internet Control Message Protocol
\end{center}
\vspace{.2in}

\begin{verbatim}
$ traceroute www.netbsd.org
traceroute to www.netbsd.org (204.152.190.12), 64 hops max, 40 byte packets
 1  eth2-3a.core1.nav.nyc.access.net (166.84.0.1)  0.256 ms  0.165 ms 0.181 ms
 2  l3v1.nyc.access.net (166.84.66.14)  1.570 ms  1.556 ms  1.437 ms
 3  gige-g3-3.core1.nyc4.he.net (209.51.171.25)  4.963 ms  2.422 ms  1.457 ms
 4  10gigabitethernet2-3.core1.ash1.he.net (72.52.92.86)  8.423 ms  8.769 ms  7.683 ms
 5  10gigabitethernet1-2.core1.atl1.he.net (184.105.213.110)  21.898 ms 19.647 ms  19.838 ms
 6  isc.gige-g2-1.core1.atl1.he.net (216.66.0.50)  77.465 ms  77.921 ms 80.519 ms
 7  iana.r1.atl1.isc.org (199.6.12.1)  77.302 ms  78.230 ms  81.782 ms
 8  int-0-5-0-1.r1.pao1.isc.org (149.20.65.37)  81.860 ms  83.780 ms 84.160 ms
 9  int-0-0-1-0.r1.sql1.isc.org (149.20.65.10)  81.543 ms  80.193 ms 84.434 ms
10  www.netbsd.org (204.152.190.12)  81.986 ms  81.008 ms  82.604 ms
$
\end{verbatim}

\subsection{TCP/IP Basics: ICMP}
\begin{center}
Internet Control Message Protocol
\end{center}

\begin{verbatim}
IP (tos 0x0, ttl 1, id 44866, offset 0, flags [none], proto UDP (17), length 40)
    166.84.7.99.44865 > 149.20.53.86.33435: [udp sum ok] UDP, length 12
IP (tos 0xc0, ttl 64, id 48796, offset 0, flags [none], proto ICMP (1), length 68)
    166.84.0.1 > 166.84.7.99: ICMP time exceeded in-transit, length 48
IP (tos 0x0, ttl 2, id 44869, offset 0, flags [none], proto UDP (17), length 40)
    166.84.7.99.44865 > 149.20.53.86.33438: [udp sum ok] UDP, length 12
IP (tos 0x0, ttl 3, id 44872, offset 0, flags [none], proto UDP (17), length 40)
    166.84.7.99.44865 > 149.20.53.86.33441: [udp sum ok] UDP, length 12
IP (tos 0x0, ttl 4, id 44875, offset 0, flags [none], proto UDP (17), length 40)
    166.84.7.99.44865 > 149.20.53.86.33444: [udp sum ok] UDP, length 12
IP (tos 0x0, ttl 252, id 6760, offset 0, flags [none], proto ICMP (1), length 56)
    154.24.25.109 > 166.84.7.99: ICMP time exceeded in-transit, length 36
...
IP (tos 0x0, ttl 248, id 0, offset 0, flags [none], proto ICMP (1), length 56)
    149.20.53.86 > 166.84.7.99: ICMP 149.20.53.86 udp port 33482 unreachable, length 36
\end{verbatim}


\subsection{TCP/IP Basics: ICMP6}
\begin{center}
Internet Control Message Protocol for IPv6
\end{center}
\vspace{.2in}

\begin{verbatim}
$ traceroute6 www.netbsd.org
traceroute6 to www.netbsd.org (2001:4f8:3:7:2e0:81ff:fe52:9a6b) from
    2001:470:30:84:204:d7b0:0:1, 64 hops max, 12 byte packets
 1  router.vc.panix.com  0.271 ms  0.282 ms  0.155 ms
 2  2001:470:30::a654:420e  5.459 ms  1.251 ms  1.073 ms
 3  gige-g3-3.core1.nyc4.he.net  1.288 ms  2.001 ms  10.176 ms
 4  10gigabitethernet8-3.core1.chi1.he.net  26.603 ms  20.532 ms  25.029 ms
 5  2001:470:1:34::2  72.033 ms  72.377 ms  72.686 ms
 6  iana.r1.ord1.isc.org  76.288 ms  72.773 ms  71.481 ms
 7  int-0-0-1-8.r1.pao1.isc.org  73.027 ms  76.489 ms  77.507 ms
 8  int-0-0-1-0.r2.sql1.isc.org  73.555 ms  75.367 ms  74.769 ms
 9  www.NetBSD.org  72.036 ms  72.522 ms  71.39 ms
$
\end{verbatim}

\subsection{TCP/IP Basics: ICMP6}
\begin{center}
Internet Control Message Protocol for IPv6
\end{center}

\begin{verbatim}
IP6 2001:470:30:84:204:d7b0:0:1.51749 >
        2001:4f8:3:7:2e0:81ff:fe52:9a6b.33435: UDP, length 12
IP6 2001:470:30:84::3 > 2001:470:30:84:204:d7b0:0:1:
        ICMP6, time exceeded in-transit [|icmp6]
IP6 2001:470:30:84:204:d7b0:0:1.51749 >
        2001:4f8:3:7:2e0:81ff:fe52:9a6b.33436: UDP, length 12
[...]
IP6 2001:470:30:84:204:d7b0:0:1.51749 >
        2001:4f8:3:7:2e0:81ff:fe52:9a6b.33461: UDP, length 12
IP6 2001:4f8:3:7:2e0:81ff:fe52:9a6b >
        2001:470:30:84:204:d7b0:0:1: ICMP6, destination unreachable[|icmp6]
\end{verbatim}

\subsection{TCP/IP Basics: TCP}
\begin{center}
Transmission Control Protocol
\end{center}
\vspace{.2in}
\begin{verbatim}
$ telnet www.yahoo.com 80
Trying 98.138.219.232...
Connected to atsv2-fp-shed.wg1.b.yahoo.com.
Escape character is '^]'.
HEAD / HTTP/1.0
\end{verbatim}

\subsection{TCP/IP Basics: TCP}
\begin{verbatim}
IP 10.183.114.37.45403 > 72.30.35.9.80: Flags [S], seq 5028151, win 65535,
	options [mss 1460,nop,wscale 6,sackOK,TS val 598758437 ecr 0], length 0
IP 72.30.35.9.80 > 10.183.114.37.45403: Flags [S.], seq 1983855029, ack 5028152, win 14480,
	options [mss 1460,sackOK,TS val 1888595968 ecr 598758437,nop,wscale 8], length 0
IP 10.183.114.37.45403 > 72.30.35.9.80: Flags [.], ack 1, win 1026,
	options [nop,nop,TS val 598758465 ecr 1888595968], length 0
IP 10.183.114.37.45403 > 72.30.35.9.80: Flags [P.], seq 1:18, ack 1, win 1026,
	options [nop,nop,TS val 598762755 ecr 1888595968], length 17: HTTP: HEAD / HTTP/1.0
[...]
IP 72.30.35.9.80 > 10.183.114.37.45403: Flags [F.], seq 325, ack 20, win 57,
	options [nop,nop,TS val 1888601741 ecr 598764209], length 0
IP 10.183.114.37.45403 > 72.30.35.9.80: Flags [F.], seq 20, ack 326, win 1026,
	options [nop,nop,TS val 598764236 ecr 1888601741], length 0
IP 72.30.35.9.80 > 10.183.114.37.45403: Flags [.], ack 21, win 57,
	options [nop,nop,TS val 1888601768 ecr 598764236], length 0
\end{verbatim}
Performance considerations? (Green Team)

\subsection{TCP/IP Basics: TCP}
\begin{center}
Transmission Control Protocol over IPv6
\end{center}
\vspace{.2in}
\begin{verbatim}
$ telnet www.netbsd.org 80
Trying 2001:4f8:3:7:2e0:81ff:fe52:9a6b...
Connected to www.netbsd.org.
Escape character is '^]'.
GET / HTTP/1.0


\end{verbatim}

\subsection{TCP/IP Basics: TCP}
\begin{center}
Transmission Control Protocol IPv6
\end{center}
\vspace{.2in}
\begin{verbatim}
IP6 2001:470:30:84:204:d7b0:0:1.58334 >
      2001:4f8:3:7:2e0:81ff:fe52:9a6b.80: S 3232473102:3232473102(0)
      win 32768 <mss 1440,nop,wscale3,sackOK,nop,nop,nop,nop,timestamp 1[|tcp]>
IP6 2001:4f8:3:7:2e0:81ff:fe52:9a6b.80 >
      2001:470:30:84:204:d7b0:0:1.58334: S 4139493123:4139493123(0)
      ack 3232473103 win 32768
IP6 2001:470:30:84:204:d7b0:0:1.58334 >
      2001:4f8:3:7:2e0:81ff:fe52:9a6b.80: . ack 1 win 4140
IP6 2001:470:30:84:204:d7b0:0:1.58334 >
      2001:4f8:3:7:2e0:81ff:fe52:9a6b.80: P 1:17(16) ack 1 win 4140
IP6 2001:4f8:3:7:2e0:81ff:fe52:9a6b.80 >
      2001:470:30:84:204:d7b0:0:1.58334: . ack 17 win 33120
\end{verbatim}


\subsection{TCP/IP Basics: UDP}
\begin{center}
User Datagram Protocol
\end{center}
\vspace{.2in}
\begin{verbatim}
$ nslookup www.yahoo.com
Server:		155.246.1.20
Address:	155.246.1.20#53

Non-authoritative answer:
www.yahoo.com        canonical name = fp3.wg1.b.yahoo.com.
fp3.wg1.b.yahoo.com       canonical name = any-fp3-lfb.wa1.b.yahoo.com.
any-fp3-lfb.wa1.b.yahoo.com        canonical name = any-fp3-real.wa1.b.yahoo.com.
Name:	any-fp3-real.wa1.b.yahoo.com
Address: 98.139.183.24

$
\end{verbatim}

\subsection{TCP/IP Basics: UDP}
\begin{center}
User Datagram Protocol
\end{center}
\vspace{.2in}
\begin{verbatim}
IP (tos 0x0, ttl 64, id 0, offset 0, flags [none],
    proto UDP (17), length 59) panix.netmeister.org.49164 >
        cache2.ns.access.net.domain: 28557+ A? www.yahoo.com. (31)

IP (tos 0x0, ttl 63, id 1862, offset 0, flags [none],
    proto UDP (17), length 207) cache2.ns.access.net.domain >
        panix.netmeister.org.49164: 28557 4/2/2
            www.yahoo.com. CNAME fp3.wg1.b.yahoo.com.[|domain]
\end{verbatim}

\subsection{TCP/IP Basics: UDP}
\begin{center}
User Datagram Protocol over IPv6
\end{center}
\vspace{.2in}
\begin{verbatim}
$ dig -6 @2001:470:20::2 www.yahoo.com

;; ANSWER SECTION:
www.yahoo.com.          300     IN      CNAME   fp3.wg1.b.yahoo.com.
fp3.wg1.b.yahoo.com.    60      IN      CNAME   any-fp3-lfb.wa1.b.yahoo.com.
any-fp3-lfb.wa1.b.yahoo.com. 300 IN     CNAME   any-fp3-real.wa1.b.yahoo.com.
any-fp3-real.wa1.b.yahoo.com. 60 IN     A       98.139.183.24

;; Query time: 51 msec
;; SERVER: 2001:470:20::2#53(2001:470:20::2)
;; WHEN: Sat Mar  3 22:49:44 2012
;; MSG SIZE  rcvd: 128

\end{verbatim}

\subsection{TCP/IP Basics: UDP}
\begin{center}
User Datagram Protocol over IPv6
\end{center}
\vspace{.2in}
\begin{verbatim}
IP6 (hlim 64, next-header: UDP (17), length: 39)
       2001:470:30:84:204:d7b0:0:1.65037 > 2001:470:20::2.53:
       [udp sum ok] 18545+ A? www.yahoo.com. (31)

IP6 (hlim 61, next-header: UDP (17), length: 119)
       2001:470:20::2.53 > 2001:470:30:84:204:d7b0:0:1.65037:
       18545 4/0/0 www.yahoo.com.[|domain]

\end{verbatim}

\subsection{TCP/IP Basics: Putting it all together}
\vspace*{\fill}
\begin{center}
	\includegraphics[scale=0.6]{pics/tcpip-stack.eps}
\end{center}
\vspace*{\fill}

\subsection{Networking}
\vspace*{\fill}
\begin{center}
	\includegraphics[scale=0.8]{pics/dsr.eps} \\
\end{center}
\vspace*{\fill}

\subsection{Homework}

\verb+https://www.cs.stevens.edu/~jschauma/615/s19-hw3.html+

\subsection{Reading}
\begin{itemize}
	\item \verb+tcpdump(8)+
	\item \verb+ktrace(1)+ / \verb+strace(1)+
	\item \verb+tcp(4)+/\verb+ip(4)+
	\item \verb+netstat(1)+
	\item \verb+nslookup(1)+
\end{itemize}
\vspace{.5in}
\begin{itemize}
	\item \verb+https://is.gd/mrrdYc+
\end{itemize}

\end{document}
