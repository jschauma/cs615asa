\documentclass[xga]{xdvislides}
\usepackage[landscape]{geometry}
\usepackage{array}
\usepackage{graphics}
\usepackage{graphicx}
\usepackage{colordvi}
\usepackage{tabularx}
\usepackage{multirow}
\usepackage{xcolor}

\begin{document}
\setfontphv

%%% Headers and footers
\lhead{\slidetitle}				% default:\lhead{\slidetitle}
\chead{CS615 - Aspects of System Administration}% default:\chead{\relax}
\rhead{Slide \thepage}				% default:\rhead{\sectiontitle}
\lfoot{\Gray{Lecture 02: Storage Models and Disks}}% default:\lfoot{\slideauthor}
\cfoot{\relax}					% default:\cfoot{\relax}
\rfoot{\Gray{\today}}

\vspace*{\fill}
\begin{center}
	\Hugesize
		CS615 - Aspects of System Administration\\ [1em]
		Storage Models and Disks\\ [1em]
	\hspace*{5mm}\blueline\\ [1em]
	\Normalsize
		Department of Computer Science\\
		Stevens Institute of Technology\\
		Jan Schaumann\\
		\verb+jschauma@stevens.edu+ \\
		\verb+https://stevens.netmeister.org/615/+
\end{center}
\vspace*{\fill}
%\setcounter{page}{0}
%\clearpage

\subsection{Team Missions}

\begin{itemize}
	\item try to get your link by Sunday night
	\item identify type of article (e.g., research paper, news, study, experience report, war story, ...)
	\item identify promotional vs. informational, independent vs. paid content, publication date
	\item make sure to relate the link to the topic and your mission
	\item avoid attachments in favor of links
\end{itemize}

\vfill
\begin{center}
	\includegraphics[scale=0.1]{pics/team-drives.eps} \\
\end{center}
\vfill

\subsection{Current Events}

CVE-2019-18634: buffer overflow in sudo when pwfeedback is enabled \\
\vspace{.125in}

{\tt https://www.sudo.ws/alerts/pwfeedback.html} \\
{\tt https://www.openwall.com/lists/oss-security/2020/01/30/6} \\
{\tt https://www.sudo.ws/repos/sudo/rev/84640592b0ff} \\

\vspace{.5in}
Try to recreate, then try to exploit.


\subsection{Staying up to date}

\begin{itemize}
        \item {\tt https://www.devopsweekly.com/}
        \item {\tt https://sreweekly.com/}
        \item {\tt https://www.nanog.org/}
        \item {\tt https://puck.nether.net/mailman/listinfo/outages}
	\item {\tt https://www.oreilly.com/webops-perf/newsletter.html}
	\item @nixCraft {\tt https://is.gd/cKEpWc} 
	\item {\tt https://www.cronweekly.com/}
\end{itemize}

\newpage
%\subsection{Let's review HW1}
%\vspace{.5in}
%Start all assignments on {\tt
%linux-lab.cs.stevens.edu}! \\
%\vspace{.5in}
%Submit plain text. \\
%\vspace{.5in}
%No resubmissions.
%
%\subsection{Let's review HW1}
%\vspace{.5in}

\subsection{Let's review HW1}
\vspace{.5in}
Running an instance:
\begin{verbatim}
$ aws ec2 run-instances --instance-type t1.micro \
        --key-name stevens --image-id ami-569ed93c
\end{verbatim}

\subsection{Let's review HW1}
\vspace{.5in}

Save yourself some typing:
\begin{verbatim}
$ alias instance='aws ec2 run-instances --key-name stevens --image-id'
$ instance ami-569ed93c --instance-type t1.micro
\end{verbatim}

\subsection{Let's review HW1}
\vspace{.5in}

Make it permanent:
\begin{verbatim}
$ echo "alias instance='aws ec2 run-instances --key-name stevens --image-id'" \
        >> ~/.bashrc
$ . ~/.bashrc
$ alias
alias instance='aws ec2 run-instances --key-name stevens --image-id'
$ 
\end{verbatim}

\subsection{Let's review HW1}
\begin{verbatim}
$ alias
alias console='aws ec2 get-console-output --instance-id'
alias ec2wait='aws ec2 wait instance-running --instance-id'
alias instance='aws ec2 run-instances --key-name stevens --image-id'
alias instances='aws ec2 describe-instances'
alias running='instances --query Reservations[*].Instances[*].[InstanceId] \
        --filters Name=instance-state-name,Values=running"
alias start-debian='startInstance ami-0dedf6a6502877301 --instance-type t2.micro'
alias start-fedora='startInstance ami-0fcbe88944a53b4c8 --instance-type t1.micro'
alias start-freebsd='startInstance ami-0de268ac2498ba33d'
alias start-netbsd='startInstance ami-569ed93c --instance-type t1.micro'
alias start-netbsd-euw1='aws ec2 run-instances --key-name stevens-euw1 \
        --image-id ami-a460c5d7 --security-group-id sg-4e947833 --region eu-west-1'
alias start-omnios='startInstance ami-0a01a5636f3c4f21c --instance-type t1.micro'
alias start-ubuntu='startInstance ami-08bc77a2c7eb2b1da --instance-type t1.micro'
alias term-instances='aws ec2 terminate-instances --instance-ids'
$ 
\end{verbatim}


\subsection{Let's review HW1}
\vspace{.5in}

ssh to an instance:
\begin{verbatim}
$ ssh -i ~/.ssh/ec2 root@<mumble>.compute-1.amazonaws.com
\end{verbatim}

\subsection{Let's review HW1}
\vspace{.5in}

Let's save ourselves some typing:
\begin{verbatim}
$ cat >>~/.ssh/config <<EOF
> Host *.amazonaws.com
>         IdentityFile ~/.ssh/ec2
>         User         root
> EOF
$ ssh <mumble>.compute-1.amazonaws.com
\end{verbatim}

\subsection{Let's review HW1}
\vspace{.5in}

How do we know what host we're connecting to?
\begin{verbatim}
$ ssh ec2-3-85-193-42.compute-1.amazonaws.com
The authenticity of host 'ec2-3-85-193-42.compute-1.amazonaws.com
(3.85.193.42)' can't be established.
ECDSA key fingerprint is SHA256:lUB2XhoRshItvOVqrYO5Lo1Mqmz69DUP++GH3Yy1P0I.
Are you sure you want to continue connecting (yes/no)? 
\end{verbatim}

\subsection{Let's review HW1}
\begin{verbatim}
$ aws ec2 get-console-output --instance-id i-0aac317689367a7b8
[...]
ec2: ###########################################################
ec2: -----BEGIN SSH HOST KEY FINGERPRINTS-----
ec2: 1024 SHA256:z5n5cOPe0Kxhrw2Uxj6SY/kwTSk5IEZtveCMvUoBwHc root@ip-10-141-117-253.ec2.internal (DSA)
ec2: 521 SHA256:lUB2XhoRshItvOVqrYO5Lo1Mqmz69DUP++GH3Yy1P0I root@ip-10-141-117-253.ec2.internal (ECDSA)
ec2: 256 SHA256:zENF+3vI3WWJC3iutGYypF3bKFCcJuTTmdDBRJJ9S14 root@ip-10-141-117-253.ec2.internal (ED25519)
ec2: 2048 SHA256:pICqhhQyGFsjnw+TosFTY8cXcif2G9R+MxDxrhgaNdk root@ip-10-141-117-253.ec2.internal (RSA)
ec2: -----END SSH HOST KEY FINGERPRINTS-----
ec2: ###########################################################
\end{verbatim}


\subsection{Let's review HW1}
How do we know what host we're connecting to?
\begin{verbatim}
$ aws ec2 get-console-output --instance-id i-0aac317689367a7b8 | grep "(ECDSA)"
ec2: 521 SHA256:lUB2XhoRshItvOVqrYO5Lo1Mqmz69DUP++GH3Yy1P0I root@ip-10-141-117-253.ec2.internal (ECDSA)
$ ssh ec2-3-85-193-42.compute-1.amazonaws.com
The authenticity of host 'ec2-3-85-193-42.compute-1.amazonaws.com
(3.85.193.42)' can't be established.
ECDSA key fingerprint is SHA256:lUB2XhoRshItvOVqrYO5Lo1Mqmz69DUP++GH3Yy1P0I.
Are you sure you want to continue connecting (yes/no)? yes
Warning: Permanently added 'ec2-3-85-193-42.compute-1.amazonaws.com,3.85.193.42' (ECDSA) to the list of known hosts.
NetBSD 7.0 (XEN3PAE_DOMU.201509250726Z)
Welcome to NetBSD - Amazon EC2 image!

[...]
We recommend that you create a non-root account and
use su(1) for root access.
ip-10-141-117-253#

\end{verbatim}

\subsection{Let's review HW1}
\begin{verbatim}
# uname -a
NetBSD ip-10-141-117-253.ec2.internal 7.0 NetBSD 7.0 (XEN3PAE_DOMU.201509250726Z) i386
\end{verbatim}

\subsection{Let's review HW1}
\begin{verbatim}
ip-10-141-117-253# ifconfig -a
xennet0: flags=8863<UP,BROADCAST,NOTRAILERS,RUNNING,SIMPLEX,MULTICAST> mtu 1500
        capabilities=2800<TCP4CSUM_Tx,UDP4CSUM_Tx>
        enabled=0
        address: 22:00:0a:8d:75:fd
        inet 10.141.117.253 netmask 0xffffff00 broadcast 10.141.117.255
        inet6 fe80::531e:e93c:d231:2b50%xennet0 prefixlen 64 scopeid 0x1
lo0: flags=8049<UP,LOOPBACK,RUNNING,MULTICAST> mtu 33184
        inet 127.0.0.1 netmask 0xff000000
        inet6 ::1 prefixlen 128
        inet6 fe80::1%lo0 prefixlen 64 scopeid 0x2
\end{verbatim}


\subsection{Let's review HW1}
\begin{verbatim}
Active Internet connections (including servers)
Proto Recv-Q Send-Q  Local Address          Foreign Address        State
tcp        0      0  10.141.117.253.22      207.172.174.21.38764   ESTABLISHED
tcp        0      0  *.22                   *.*                    LISTEN
udp        0      0  *.68                   *.*                   
Active Internet6 connections (including servers)
Proto Recv-Q Send-Q  Local Address          Foreign Address        (state)
tcp6       0      0  *.22                   *.*                    LISTEN
Active UNIX domain sockets
Address  Type   Recv-Q Send-Q    Inode     Conn       Refs  Nextref Addr
c1dc3378 stream      0      0        0 c1dc33c8          0        0 
[...]
\end{verbatim}

\subsection{Let's review HW1}
\begin{verbatim}
ip-10-141-117-253# df -hi
Filesystem         Size       Used      Avail %Cap  iUsed   iAvail %iCap Mounted on
/dev/xbd1a         959M       485M       426M  53%  23240   507254   4%  /
/dev/xbd0a         246M       2.2M       231M   0%     15    65521   0%  /grub
kernfs             1.0K       1.0K         0B 100%      0        0   0%  /kern
ptyfs              1.0K       1.0K         0B 100%      0        0   0%  /dev/pts
procfs             4.0K       4.0K         0B 100%     15      517   2%  /proc
ip-10-141-117-253# mount
/dev/xbd1a on / type ffs (local)
/dev/xbd0a on /grub type ext2fs (local)
kernfs on /kern type kernfs (local)
ptyfs on /dev/pts type ptyfs (local)
procfs on /proc type procfs (local)
\end{verbatim}


\subsection{Let's review HW1}
\small
\begin{verbatim}
# fdisk /dev/xbd0
fdisk: primary partition table invalid, no magic in sector 0
fdisk: Cannot determine the number of heads
Disk: /dev/xbd0d
NetBSD disklabel disk geometry:
cylinders: 1024, heads: 1, sectors/track: 2048 (2048 sectors/cylinder)
total sectors: 2097152, bytes/sector: 512

BIOS disk geometry:
cylinders: 130, heads: 255, sectors/track: 63 (16065 sectors/cylinder)
total sectors: 2097152

Partitions aligned to 16065 sector boundaries, offset 63

Partition table:
0: <UNUSED>
1: <UNUSED>
2: <UNUSED>
3: <UNUSED>
Bootselector disabled.
No active partition.
Drive serial number: 0 (0x00000000)
\end{verbatim}
\Normalsize

\subsection{Let's review HW1}
\small
\begin{verbatim}
# disklabel /dev/rxbd0
# /dev/rxbd0d:
type: ESDI
disk: Xen Virtual ESDI
label: fictitious
flags:
bytes/sector: 512
sectors/track: 2048
tracks/cylinder: 1
sectors/cylinder: 2048
cylinders: 1024
total sectors: 2097152
rpm: 3600
interleave: 1
trackskew: 0
cylinderskew: 0
headswitch: 0           # microseconds
track-to-track seek: 0  # microseconds
drivedata: 0 

4 partitions:
#        size    offset     fstype [fsize bsize cpg/sgs]
 a:   2097152         0     4.2BSD      0     0     0 # (Cyl.      0 -   1023)
 d:   2097152         0     unused      0     0       # (Cyl.      0 -   1023)
disklabel: boot block size 0
disklabel: super block size 0
\end{verbatim}
\Normalsize

\subsection{Storage Models and Disks}
\begin{verbatim}
$ ssh linux-lab.cs.stevens.edu
$ df -hT
\end{verbatim}

\begin{verbatim}
$ dd if=/dev/zero of=/tmp/big bs=1G
[...]
\end{verbatim}

Now try to {\tt ssh} to that host...

\subsection{Storage Models and Disks}

File sizes are not always what they seem to be.

\begin{verbatim}
$ mkdir /tmp/${USER}
$ export LARGE=/tmp/${USER}/large
$ truncate -s $(df /tmp | awk '/^\// { print $4; }')0000 ${LARGE}
$ ls -l ${LARGE}
$ du ${LARGE}
$ stat ${LARGE}
$ cp ${LARGE} ${LARGE}2
$ du ${LARGE}2
$ cat ${LARGE} > ${LARGE}2
$ ls -l ${LARGE}*
$ du ${LARGE}*
\end{verbatim}

\subsection{Storage Models and Disks}
How many files can be created on {\tt /tmp}?

\begin{verbatim}
$ ssh linux-lab.cs.stevens.edu
$ df -i /tmp
$ rm /tmp/${USER}/large2
$ df -i /tmp
\end{verbatim}

\subsection{Storage Models and Disks}

\begin{verbatim}
$ ssh linux-lab.cs.stevens.edu
$ cd /tmp
$ df -i /tmp
$ touch newfile
$ cc -Wall ~jschauma/tmp/mkfiles.c
$ ./a.out /tmp/${USER}
$ ls -ld /tmp/${USER}
$ ls /tmp/${USER} | wc -l
$ touch newerfile
$ echo "hello hello hello" >> newfile
$ rm -fr /tmp/${USER}
\end{verbatim}

See also:
\verb+https://is.gd/nX07RR+

\subsection{Storage Models and Disks}
Important lessons: \\

File sizes are not always what they seem to be. \\

Error messages aren't always what they seem to be! \\

All resources are finite.

\subsection{Storage Models and Disks}
\begin{itemize}
	\item basic disk concepts
	\item basic filesystem concepts
	\item file systems (next class)
\end{itemize}

\subsection{Topics covered}
\begin{itemize}
	\item basic disk concepts
		\begin{itemize}
			\item storage models
			\item (disk interfaces)
			\item physical disk structure
			\item partitions
		\end{itemize}
	\item basic filesystem concepts
\end{itemize}

\subsection{Topics covered}
\begin{itemize}
	\item basic disk concepts
		\begin{itemize}
			\item storage models
			\item (disk interfaces)
			\item physical disk structure
			\item partitions
		\end{itemize}
	\item basic filesystem concepts
		\begin{itemize}
			\item RAID
			\item logical volume managment
			\item device formatting
		\end{itemize}
\end{itemize}

\newpage
\vspace*{\fill}
\begin{center}
	\Hugesize
		Basic Disk Concepts \\ [1em]
	\hspace*{5mm}
	\blueline\\
	\hspace*{5mm}\\
		Storage Models
\end{center}
\vspace*{\fill}

\subsection{Basic Disk Concepts: Storage Models}
Direct Attached Storage (DAS)
\vfill
\begin{center}
	\includegraphics[scale=0.8]{pics/das.eps} \\
\end{center}
\verb+ssh lab 'df -hiT /'+
\vfill

\subsection{Team Missions}
\vfill
\begin{center}
	\includegraphics[scale=0.2]{pics/team-drives.eps} \\
\end{center}
{\bf Black Team} \\
Certificable File System Correctness \\
\vfill

\subsection{Basic Disk Concepts: Storage Models}
Network Attached Storage (NAS)
\vfill
\begin{center}
	\includegraphics[scale=0.5]{pics/nas.eps} \\
\end{center}
\verb+ssh lab 'df -hiT /home/$(whoami)'+
\vfill

\subsection{Team Missions}
\vfill
\begin{center}
	\includegraphics[scale=0.2]{pics/team-drives.eps} \\
\end{center}
\hspace*{\fill}{\bf \textcolor{red}{Red Team}} \\
\hspace*{\fill}{\tt https://is.gd/cmjmWu} \\
\hspace*{\fill}Linux Privilege Escalation using Misconfigured NFS
\vfill

\subsection{Basic Disk Concepts: Storage Models}
Storage Area Networks (SAN)
\vfill
\begin{center}
	\includegraphics[scale=0.5]{pics/san-nas-das.eps} \\
\end{center}
\vfill

\subsection{Basic Disk Concepts: Storage Models}
Cloud Storage (Examples: EBS, S3)
\vfill
\begin{center}
	\includegraphics[scale=0.6]{pics/cloud-storage.eps} \\
\end{center}
\vfill

\subsection{Basic Disk Concepts: Storage Models: Cloud Storage}
\begin{verbatim}
$ aws ec2 describe-instances
[...]
/dev/sda1	ebs	None	paravirtual
BLOCKDEVICEMAPPINGS	/dev/sda
EBS	2014-01-25T20:18:19.000Z	True	attached vol-a0d000d6
[...]
\end{verbatim}

\subsection{Team Missions}
\vfill
\begin{center}
	\includegraphics[scale=0.2]{pics/team-drives.eps} \\
\end{center}
\hspace*{2.5in}{\bf \textcolor{blue}{Blue Team}} \\
\hspace*{2.5in}{\tt https://is.gd/Tm6IQi} \\
\hspace*{2.5in}Azure Cloud Security Vulnerability
\vfill

\newpage
\vspace*{\fill}
\begin{center}
	\Hugesize
		Basic Disk Concepts \\ [1em]
	\hspace*{5mm}
	\blueline\\
	\hspace*{5mm}\\
		Disk Devices
\end{center}
\vspace*{\fill}


\subsection{Basic Disk Concepts: Disk Devices}
	\begin{center}
		\includegraphics[scale=0.9]{pics/ide-drive.eps} \\
	\end{center}

\subsection{Basic Disk Concepts: Disk Devices}
Security affects everything. \\
\begin{center}
	\includegraphics[scale=0.9]{pics/equation-tweet.eps} \\
	\verb+https://is.gd/bK0rwd+
\end{center}

\subsection{Basic Disk Concepts: Disk Devices}
	\begin{center}
		\includegraphics[scale=0.5]{pics/ssd.eps} \\
	\end{center}

\subsection{Basic Disk Concepts: Disk Devices}
	\begin{center}
		\includegraphics[scale=0.5]{pics/ddrdrive.eps} \\
	\end{center}

\subsection{Basic Disk Concepts: Disk Devices}
\begin{figure}[hb]
	\begin{center}
		\includegraphics[scale=0.5]{pics/jbod-inside.eps}
		\hspace*{15mm}
		\includegraphics[scale=0.5]{pics/jbod-front.eps} \\
	\end{center}
\end{figure}


\subsection{Basic Disk Concepts: Disk Devices}
	\begin{center}
		\includegraphics[scale=0.9]{pics/xraid.eps} \\
	\end{center}


\subsection{Basic Disk Concepts: Disk Devices}
	\begin{center}
		\includegraphics[scale=0.7]{pics/netapp.eps} \\
	\end{center}


\newpage
\vspace*{\fill}
\begin{center}
    \Hugesize
        Hooray! \\ [1em]
    \hspace*{5mm}
    \blueline\\
    \hspace*{5mm}\\
        5 Minute Break
\end{center}
\vspace*{\fill}

\newpage
\vspace*{\fill}
\begin{center}
	\Hugesize
		Basic Disk Concepts\\ [1em]
	\hspace*{5mm}
	\blueline\\
	\hspace*{5mm}\\
		Physical Disk Structure
\end{center}
\vspace*{\fill}

\subsection{Basic Disk Concepts: Disk Devices}
\vfill
	\begin{center}
		\includegraphics[scale=1.2]{pics/satavide.eps} \\
	\end{center}
\vfill


\subsection{Basic Disk Concepts: Disk Devices}
%\begin{figure}[hb]
	\begin{center}
		\includegraphics[scale=0.9]{pics/busted-disk.eps} \\
	\end{center}
%\end{figure}

\subsection{Basic Disk Concepts: Disk Devices}
\vfill
	\begin{center}
		\includegraphics[scale=0.9]{pics/6platter.eps} \\
	\end{center}
\vfill

\subsection{Basic Disk Concepts: Physical Disk Structure}
\vfill
	\begin{center}
		\includegraphics[scale=1.2]{pics/cylinders.eps} \\
	\end{center}
\vfill

\subsection{Basic Disk Concepts: Physical Disk Structure}
Hard drive performance determined by:
\begin{itemize}
	\item seek time
	\item rotational latency
	\item internal data rate
	\item a few other negligible factors (external data rate, command
		overhead, access time, etc.)
\end{itemize}


\newpage
\vspace*{\fill}
\begin{center}
	\Hugesize
		Basic Disk Concepts\\ [1em]
	\hspace*{5mm}
	\blueline\\
	\hspace*{5mm}\\
		Partitions
\end{center}
\vspace*{\fill}

\subsection{Basic Disk Concepts: Partitions}
	\begin{center}
		\includegraphics[scale=0.65]{pics/chs.eps}
		\includegraphics[scale=0.7]{pics/disk.partition.eps} \\
	\end{center}



\subsection{Basic Disk Concepts: Partitions}
NetBSD example (from {\tt disklabel(8)})

\begin{tabular}{ l l c }
Partition 'a': & / & \\
Partition 'b': & swap & \\
Partition 'e': & /home & \\
\end{tabular}

\begin{verbatim}
#        size    offset   fstype [fsize bsize cpg/sgs]
a:  20972385        63   4.2BSD   4096 32768  1180  # (Cyl.      0*- 20805)
b:   1048320  20972448     swap                     # (Cyl.  20806 - 21845)
c:  78140097        63   unused      0     0        # (Cyl.      0*- 77519)
d:  78140160         0   unused      0     0        # (Cyl.      0 - 77519)
e:  56119392  22020768   4.2BSD   4096 32768 58528  # (Cyl.  21846 - 77519)
\end{verbatim}

\subsection{Basic Disk Concepts: Partitions}
NetBSD example (from {\tt disklabel(8)})

\begin{tabular}{ l l c }
Partition 'a': & / & 10 GB\\
Partition 'b': & swap & \\
Partition 'e': & /home & 26 GB\\
\end{tabular}

\begin{verbatim}
#        size    offset   fstype [fsize bsize cpg/sgs]
a:  20972385        63   4.2BSD   4096 32768  1180  # (Cyl.      0*- 20805)
b:   1048320  20972448     swap                     # (Cyl.  20806 - 21845)
c:  78140097        63   unused      0     0        # (Cyl.      0*- 77519)
d:  78140160         0   unused      0     0        # (Cyl.      0 - 77519)
e:  56119392  22020768   4.2BSD   4096 32768 58528  # (Cyl.  21846 - 77519)
\end{verbatim}


\subsection{Basic Disk Concepts: Partitions}
Solaris example (from {\tt format(1m)}):
\begin{verbatim}
Current partition table (original):
Total disk cylinders available: 38758 + 2 (reserved cylinders)

Part      Tag    Flag     Cylinders         Size            Blocks
  0       root    wm       3 -  3764        3.62GB    (3762/0/0)   7584192
  1       swap    wu    3765 -  4364      590.62MB    (600/0/0)    1209600
  2     backup    wm       0 - 38757       37.26GB    (38758/0/0) 78136128
  3 unassigned    wm       0                0         (0/0/0)            0
  4 unassigned    wm       0                0         (0/0/0)            0
  5 unassigned    wm       0                0         (0/0/0)            0
  6 unassigned    wm       0                0         (0/0/0)            0
  7       home    wm    4365 - 38757       33.06GB    (34393/0/0) 69336288
  8       boot    wu       0 -     0        0.98MB    (1/0/0)         2016
  9 alternates    wu       1 -     2        1.97MB    (2/0/0)         4032
\end{verbatim}

\subsection{Basic Disk Concepts: Partitions}
Linux example (from {\tt fdisk(8)}):
\begin{verbatim}
Disk /dev/sda: 80.0 GB, 80000000000 bytes
255 heads, 63 sectors/track, 9726 cylinders
Units = cylinders of 16065 * 512 = 8225280 bytes

   Device Boot      Start         End      Blocks   Id  System
/dev/sda1   *           1          33      265041   83  Linux
/dev/sda2              34        9726    77859022+  83  Linux
\end{verbatim}

\subsection{Basic Disk and Filesystem Concepts: RAID and Logical Volumes}
\begin{itemize}
	\item allow file systems to be larger than the physical size of a disk
	\item inrease I/O performance when {\em striped}
	\item fault tolerant when {\em mirrored} or {\em plexed}
\end{itemize}
\vfill
\begin{center}
        \includegraphics[scale=0.5]{pics/raid-0.eps}
        \hspace{.5in}
        \includegraphics[scale=0.5]{pics/raid-1.eps} \\
        \vspace{.2in}
        \includegraphics[scale=0.5]{pics/raid-5.eps}
\end{center}
\vfill

\subsection{Team Missions}
\vfill
\begin{center}
	\includegraphics[scale=0.2]{pics/team-drives.eps} \\
\end{center}
\vfill
\hspace*{\fill}{\bf \textcolor{green}{Green Team}} \\
\hspace*{\fill}{\tt https://is.gd/0sPSTW} \\
\hspace*{\fill}Top 4 Causes Of Storage I/O Bottlenecks

\subsection{Cloud Storage Exercise}
{\tt https://stevens.netmeister.org/615/filesystems-exercise.html}
\\

\begin{itemize}
	\item create two OmniOS instances
	\item create a 1GB volume and attach it to one of the instances
	\item create a new filesystem on the volume and mount it
	\item create a file on the new filesystem
	\item terminate the first instance
	\item attach the volume to the second instance
	\item retrieve the file from the volume via the second instance
\end{itemize}

If time permits, repeat using a Linux instance.
\vspace{.25in}
Useful commands: \\
{\tt aws ec2 create-volume}, {\tt aws ec2 attach-volume}, {\tt
format(1M)}, {\tt newfs(1M)}, {\tt mount(8)}

\subsection{Basic Disk Concepts: Storage Models: Cloud Storage}
\begin{verbatim}
$ aws ec2 create-volume --size 1 --availability-zone us-east-1d
[...]
$ aws ec2 attach-volume --volume-id vol-9d3aeaeb --instance-id \
        i-dd74f0f3 --device /dev/sdh
[...]
$ ssh <hostname>
# format
format> fdisk
format> verify
format> label
# newfs /dev/rdsk/c1t2160d0s0
[set proper partition sizes]
# mount /dev/dsk/c1t2160d0s0 /mnt
# df -Th /mnt
[...]
# fstyp -v /dev/rdsk/c1t2160d0s0  | more
[...]
\end{verbatim}

% Cross mounting fails, but here's something that does
% work: NetBSD create filesystem, Ubuntu mount
% $ sudo mount -t ufs -o ro,ufstype=44bsd /dev/xvdh1
% /mnt

\subsection{Homework}
Repeat the examples from class.  Make sure you understand the commands and
how they relate to the concepts we discussed.  Repeat for a different OS,
for example: \\

\begin{itemize}
	\item {\tt ami-0fcbe88944a53b4c8} -- Fedora 31
	\item {\tt ami-00e61b69944d5d2a3} -- FreeBSD 12.0
	\item {\tt ami-569ed93c} -- NetBSD 7.0
	\item {\tt ami-0a01a5636f3c4f21c} -- OmniOS r151030
\end{itemize}

\vspace{.5in}
Remember to {\em shut down} your EC2 instances and to {\em delete} any unused ESB
volumes!

%\vspace{.5in}
%{\tt https://stevens.netmeister.org/615/s19-hw2.html}


\subsection{Reading}
\begin{itemize}
	\item \verb+https://is.gd/5mndwA+
	\item \verb+https://is.gd/ig4QP5+
	\item \verb+https://is.gd/9YeIKh+
\end{itemize}

\subsection{Reading}
Basic Disk Concepts:
\begin{itemize}
	\item disklabel(8), fdisk(8)
	\item format(1m)
\end{itemize}
RAID:
\begin{itemize}
	\item \verb+https://en.wikipedia.org/wiki/RAID+
\end{itemize}
Basic Filesystem Concepts:
\begin{itemize}
	\item \verb+https://is.gd/8KHnQj+
	\item \verb+https://is.gd/wGgJ0e+
	\item newfs(8)
\end{itemize}
NFS: \verb+https://is.gd/70yqMZ+

\subsection{Additional Content}

Content for topics we couldn't fit into class below.
\\

Please remember to read through {\tt
https://www.netmeister.org/book/04-file-systems.pdf}.

\newpage
\vspace*{\fill}
\begin{center}
	\Hugesize
		Basic Disk Concepts \\ [1em]
	\hspace*{5mm}
	\blueline\\
	\hspace*{5mm}\\
		Disk Interfaces
\end{center}
\vspace*{\fill}

\subsection{Basic Disk Concepts: Disk Interfaces: SCSI}
\vfill
	\begin{center}
		\includegraphics[scale=1.0]{pics/scsi-disk.eps} \\
	\end{center}
\vfill

% IDE
\subsection{Basic Disk Concepts: Disk Interfaces: ATA}
\vfill
	\begin{center}
		\includegraphics[scale=1.1]{pics/satavide.eps} \\
	\end{center}
\vfill

\subsection{Basic Disk Concepts: Disk Interfaces: ATA}
\vfill
	\begin{center}
		\includegraphics[scale=0.8]{pics/satavpata.eps} \\
	\end{center}
\vfill


% FC

\subsection{Basic Disk Concepts: Disk Interfaces: Fibre Channel}
\vfill
	\begin{center}
		\includegraphics[scale=0.9]{pics/fc-connector.eps} \\
	\end{center}
\vfill

\subsection{Basic Disk Concepts: Disk Interfaces: Fibre Channel}
\vfill
	\begin{center}
		\includegraphics[scale=1.5]{pics/fc-optical.eps} \\
	\end{center}
\vfill

\subsection{Basic Disk Concepts: Disk Interfaces: Fibre Channel}
\vfill
	\begin{center}
		\includegraphics[scale=1.0]{pics/fc-topologies.eps} \\
	\end{center}
\vfill

\subsection{Basic Disk Concepts: Disk Interfaces: Fibre Channel}
\vfill
	\begin{center}
		\includegraphics[scale=0.8]{pics/fc-switch.eps} \\
	\end{center}
\vfill

\subsection{Basic Disk Concepts: Disk Interfaces: Fibre Channel}
\vfill
	\begin{center}
		\includegraphics[scale=0.4]{pics/fc-switched.eps} \\
	\end{center}
\vfill

% SANs
\subsection{Basic Disk Concepts: Disk Interfaces: SANs}
\begin{itemize}
	\item ATA over Ethernet ({\em AoE}):
		\begin{itemize}
			\item create low-cost SAN
			\item ATA encapsulated into Ethernet frames
		\end{itemize}
	\item Fibre Channel over Ethernet ({\em FCoE}):
		\begin{itemize}
			\item consolidate IP and FC/SAN networks
			\item FC encapsulated into Ethernet frames
		\end{itemize}

	\item *oE:
		\begin{itemize}
			\item no TCP/IP overhead
			\item restricted to a single Layer 2 network
			\item no inherent security features
		\end{itemize}
	\item iSCSI
		\begin{itemize}
			\item SCSI encapsulated in TCP/IP packets
		\end{itemize}
\end{itemize}


\subsection{Reading}
Disk Interfaces:
\begin{itemize}
	\item SCSI:
		\begin{itemize}
			\item \verb+https://en.wikipedia.org/wiki/Scsi+
			\item scsi(4), scsictl(8);
		\end{itemize}
	\item ATA:
		\begin{itemize}
			\item \verb+http://www.ata-atapi.com/+
			\item \verb+https://en.wikipedia.org/wiki/Advanced_Technology_Attachment+
			\item \verb+https://en.wikipedia.org/wiki/Sata+
		\end{itemize}
\end{itemize}

\subsection{Reading}
Disk Interfaces:
\begin{itemize}
	\item Serial attached SCSI:
		\begin{itemize}
			\item \verb+https://en.wikipedia.org/wiki/Serial_attached_SCSI+
		\end{itemize}
	\item Fibre Channel:
		\begin{itemize}
			\item \verb+https://hsi.web.cern.ch/HSI/fcs/fcs.html+
			\item \verb+https://en.wikipedia.org/wiki/Fibrechannel+
		\end{itemize}
	\item AoE, FCoE, iSCSI:
		\begin{itemize}
			\item \verb+https://en.wikipedia.org/wiki/ATA_over_Ethernet+
			\item \verb+https://en.wikipedia.org/wiki/FCoE+
			\item \verb+https://en.wikipedia.org/wiki/ISCSI+
		\end{itemize}
\end{itemize}


\end{document}
