\documentclass[xga]{xdvislides}
\usepackage[landscape]{geometry}
\usepackage{graphics}
\usepackage{graphicx}
\usepackage{colordvi}
\usepackage{fancyvrb}

\fvset{fontfamily=courier,commandchars=\\\{\}}

\begin{document}
\setfontphv

%%% Headers and footers
\lhead{\slidetitle}                               % default:\lhead{\slidetitle}
\chead{CS615 - Aspects of System Administration}% default:\chead{\relax}
\rhead{Slide \thepage}                       % default:\rhead{\sectiontitle}
\lfoot{\Gray{Monitoring, Configuration Management}}% default:\lfoot{\slideauthor}
\cfoot{\relax}                               % default:\cfoot{\relax}
\rfoot{\Gray{\today}}

\newcommand{\smallish}{\fontsize{16}{16}\selectfont}

\vspace*{\fill}
\begin{center}
	\Hugesize
		CS615 - Aspects of System Administration\\ [1em]
		Monitoring, Configuration Management\\ [1em]
	\hspace*{5mm}\blueline\\ [1em]
	\Normalsize
		Department of Computer Science\\
		Stevens Institute of Technology\\
		Jan Schaumann\\
		\verb+jschauma@stevens-tech.edu+ \\
		\verb+https://stevens.netmeister.org/615/+
\end{center}
\vspace*{\fill}

\newpage
\vspace*{\fill}
\begin{center}
	\Hugesize
		Hooray!\\ [1em]
	\hspace*{5mm}
	\blueline\\
	\hspace*{5mm}\\
		5 minute break
\end{center}
\vspace*{\fill}


\subsection{Problem Report}
\vspace*{\fill}
\Huge
\begin{center}
``Something's wrong.''
\end{center}
\Normalsize
\vspace*{\fill}

\subsection{Now what?}
\begin{center}
	\includegraphics[scale=0.55]{pics/monkey.eps}
\end{center}

\subsection{Problem Report}
\vspace*{\fill}
\Huge
\begin{center}
``The system feels slow.'' \\
\addvspace{.5in}
``I can't log in.'' \\
\addvspace{.5in}
``My mail was not delivered.'' \\
\addvspace{.5in}
``The site is down.''
\end{center}
\Normalsize
\vspace*{\fill}

\subsection{Now what?}
\begin{center}
	\includegraphics[scale=0.55]{pics/monkey.eps}
\end{center}

\subsection{To the logs!}
\begin{center}
	\includegraphics[scale=0.55]{pics/logs.eps}
\end{center}

\subsection{Answers}
``The system feels slow.''
\begin{verbatim}
up 1318 days, 13:46, 1 user, load averages: 993.81, 272.91, 1012.18
\end{verbatim}

\addvspace{.3in}
``I can't log in.''
\begin{verbatim}
Apr 6 09:25:56 <auth.info>hostname sshd[1624]: Failed password for jdoe from
115.239.231.100 port 1047 ssh2
\end{verbatim}

\addvspace{.3in}
``My mail was not delivered.''
\begin{verbatim}
Apr 11 16:15:40 panix postfix/smtpd[7566]: connect from unknown[122.3.68.122]
Apr 11 16:15:41 panix postfix/smtpd[7566]: NOQUEUE: reject_warning: RCPT from
unknown[122.3.68.122]: 450 4.7.1 Client host rejected: cannot find your hostname,
[122.3.68.122]; from=<McneilRomany28@pldt.net> to=<jschauma@stevens.edu>
proto=ESMTP helo=<122.3.68.122.pldt.net>
\end{verbatim}

\subsection{Answers}
``The site is down.'' \\

\begin{verbatim}
94.242.252.41 - "" [11/Apr/2016:19:18:47 -0400] "GET /secret/ HTTP/1.1"
403 524 "-" "Mozilla/5.0 (Macintosh; Intel Mac OS X 10.9; rv:28.0)
Gecko/20100101 Firefox/28.0"
\end{verbatim}

\subsection{Answers}
``The site is down.'' \\

\begin{verbatim}
94.242.252.41 - "" [11/Apr/2016:19:18:47 -0400] "GET /secret/ HTTP/1.1"
403 524 "-" "Mozilla/5.0 (Macintosh; Intel Mac OS X 10.9; rv:28.0)
Gecko/20100101 Firefox/28.0"
\end{verbatim}

\addvspace{.2in}
\begin{center}
	\includegraphics[scale=0.25]{pics/monkey.eps}
\end{center}

\subsection{Events}
\vspace*{\fill}
\Huge
\begin{center}
``Something's wrong.'' is just an {\em unexpected} or
{\em undesirable} event.
\end{center}
\Normalsize
\vspace*{\fill}

\subsection{Events}
\vspace*{\fill}
\Huge
\begin{center}
``Something's wrong.'' is just an {\em unexpected} or
{\em undesirable} event. \\
\vspace{.4in}
{\em Events} happen all the time.
\end{center}
\Normalsize
\vspace*{\fill}

\subsection{Events}
\vspace*{\fill}
\Huge
\begin{center}
``Something's wrong.'' is just an {\em unexpected} or
{\em undesirable} event. \\
\vspace{.4in}
{\em Events} happen all the time. \\
\vspace{.4in}
Being able to identify {\em relevant} events allows
you to diagnose, predict and even prevent {\em
undesirable} events.
\end{center}
\Normalsize
\vspace*{\fill}

\subsection{Events}
\vspace*{\fill}
\Huge
\begin{center}
In order to be able to identify an event as {\em
unexpected}, you have to have {\em expected} events.
\end{center}
\Normalsize
\vspace*{\fill}

\subsection{Expected Events}
\vspace*{\fill}
\Huge
\begin{center}
Know your applications.
\end{center}
\Normalsize
\vspace*{\fill}

\subsection{Expected Events}
\vspace*{\fill}
\Huge
\begin{center}
Know your applications. \\
\vspace{.4in}
Know your users.
\end{center}
\Normalsize
\vspace*{\fill}

\subsection{Expected Events}
\vspace*{\fill}
\Huge
\begin{center}
Know your applications. \\
\vspace{.4in}
Know your users. \\
\vspace{.4in}
Know your traffic patterns.
\end{center}
\Normalsize
\vspace*{\fill}

\subsection{Expected Events}
\vspace*{\fill}
\Huge
\begin{center}
Know your applications. \\
\vspace{.4in}
Know your users. \\
\vspace{.4in}
Know your traffic patterns. \\
\vspace{.4in}
{\em Know your systems.}
\end{center}
\Normalsize
\vspace*{\fill}

\subsection{Events and Metrics}
\vspace*{\fill}
\begin{verbatim}
$ dict event
  event
      n 1: something that happens at a given place and time
      2: a special set of circumstances; "in that event, the first
         possibility is excluded"; "it may rain in which case the
         picnic will be canceled" [syn: {event}, {case}]


$ dict metric
  metric
      3: a system of related measures that facilitates the
         quantification of some particular characteristic [syn:
         {system of measurement}, {metric}]

\end{verbatim}
\vspace*{\fill}

\subsection{Events and Metrics}
\begin{center}
	\includegraphics[scale=0.75]{pics/events-metrics.eps}
\end{center}

\subsection{Events and Metrics}
Events
\begin{itemize}
	\item may occur rarely / frequently / constantly
	\item can be collected in logs
	\item may be comprised of other events
	\item may be: ’something happened’
	\item may be: ’nothing (new) happened’
\end{itemize}
\addvspace{.5in}

Metrics:
\begin{itemize}
	\item correlation of related events
	\item may help identify outliers
	\item may trigger events
	\item may help make (automated or interactive) decisions
\end{itemize}


\subsection{Collecting Data}
{\em Counters}: easy, numeric data tracking individual events. Example: HTTP status codes

\addvspace{.5in}
{\em Timers}: easy, numeric data tracking event duration. Example: Time to send all
data for a successful HTTP request.

\addvspace{.5in}
{\em Thresholds}: easy, numeric trigger for events; may itself trigger events or metrics.
Example: more than N HTTP hits in X seconds yield 404.

% % \subsection{Counting counters, timing timers...}
% % SNMP
% % \vspace{.5in}
% % \begin{center}
% % 	\Huge
% % 	A complete network management system to monitor network-attached devices.
% % \end{center}
% % \Normalsize
% % 
% % \subsection{SNMP}
% % Base concepts:
% % \begin{itemize}
% % 	\item managed devices run an {\em snmp agent} or d\ae mon
% % 	\item information about the device is exposed in {\em management information bases}
% % 	\item parts of a system are made available in {\em read-only} mode
% % 	\item parts of a system may be made available in {\em write} mode
% % 	\item certain conditions may trigger actions or {\em traps}
% % 	\item normally uses UDP 161 for the {\em agent} and 162 for the {\em manager}
% % \end{itemize}
% % 
% % \subsection{SNMP}
% % Management Information Bases (MIBs):
% % \begin{itemize}
% % 	\item hierarchical namespace
% % 	\item contains {\em Object Identifiers} (OIDs)
% % 	\item written in {\em Abstract Syntax Notation One} (ASN.1)
% % 	\item often vendor defined
% % \end{itemize}
% % 
% % \subsection{SNMP Versions}
% % SNMPv1:
% % \begin{itemize}
% % 	\item de-facto standard
% % 	\item poor security (``community strings'' act as passwords)
% % \end{itemize}
% % \vspace{.2in}
% % SNMPv2:
% % \begin{itemize}
% % 	\item improvements in the area of performance (\verb+GETBULK+ instead of \verb+GETNEXT+) and security
% % 	\item comes in the flavors {\em SNMPv2c}, {\em SNMPv1.5} and {\em SNMPv2u}
% % \end{itemize}
% % \vspace{.2in}
% % SNMPv3:
% % \begin{itemize}
% % 	\item official standard
% % 	\item adds authentication, privacy and access control
% % \end{itemize}
% % 
% % \subsection{SNMP: An example}
% % \smallish
% % \begin{verbatim}
% % $ snmpwalk -c public -v 1 bluemoon.cs.stevens-tech.edu
% % iso.3.6.1.2.1.1.1.0 = STRING: "HP ETHERNET MULTI-ENVIRONMENT,ROM none,JETDIRECT,JD147,EEPROM
% % JDI2300 0013,CIDATE 07/13/2013"
% % iso.3.6.1.2.1.1.2.0 = OID: iso.3.6.1.4.1.11.2.3.9.1
% % iso.3.6.1.2.1.1.3.0 = Timeticks: (293219400) 33 days, 22:29:54.00
% % [...]
% % iso.3.6.1.2.1.25.3.2.1.3.1 = STRING: "HP Color LaserJet CP5520 Series"
% % iso.3.6.1.2.1.25.3.2.1.3.2 = STRING: "SanDisk SDSA5AK-008G-1006"
% % [...]
% % iso.3.6.1.2.1.43.16.5.1.2.1.1 = STRING: "Sleep mode on"
% % \end{verbatim}
% % 
% % \subsection{SNMP: An example}
% % \smallish
% % \begin{verbatim}
% % $ snmpwalk -c public -v 1 gw.cc.stevens-tech.edu
% % SNMPv2-MIB::sysDescr.0 = STRING: Cisco IOS Software, s72033_rp Software
% % (s72033_rp-ADVIPSERVICESK9_WAN-M), Version 12.2(33)SXH, RELEASE SOFTWARE (fc5)
% % DISMAN-EVENT-MIB::sysUpTimeInstance = Timeticks: (3112108803) 360 days, 4:44:48.03
% % SNMPv2-MIB::sysContact.0 = STRING: chose@stevens.edu x5457
% % SNMPv2-MIB::sysName.0 = STRING: gw.cc.stevens-tech.edu
% % SNMPv2-MIB::sysLocation.0 = STRING: campus:sl:0:machineroom
% % SNMPv2-MIB::sysORLastChange.0 = Timeticks: (0) 0:00:00.00
% % [...]
% % IF-MIB::ifPhysAddress.1 = STRING: 0:17:95:68:d2:dc
% % IF-MIB::ifPhysAddress.2 = STRING: 0:18:74:1c:e3:80
% % [...]
% % IF-MIB::ifAdminStatus.1 = INTEGER: up(1)
% % IF-MIB::ifAdminStatus.2 = INTEGER: down(2)
% % [...]
% % IF-MIB::ifInOctets.1 = Counter32: 147341347
% % IF-MIB::ifInOctets.2 = Counter32: 487894092
% % [...]
% % IF-MIB::ifOutOctets.1 = Counter32: 956876160
% % IF-MIB::ifOutOctets.2 = Counter32: 1532452749
% % [...]
% % RFC1213-MIB::ipRouteDest.66.193.255.0 = IpAddress: 66.193.255.0
% % RFC1213-MIB::ipRouteDest.66.194.0.0 = IpAddress: 66.194.0.0
% % [...]
% % \end{verbatim}
% % \Normalsize
% % 
% % \subsection{SNMP: An example}
% % \smallish
% % \begin{verbatim}
% % $ snmpwalk -Os -c public -v 1 localhost
% % iso.3.6.1.2.1.1.1.0 = STRING: "Linux avatar 3.2.0-51-generic #77-Ubuntu SMP Wed Jul 24 20:18:19
% % UTC 2013 x86_64"
% % iso.3.6.1.2.1.1.2.0 = OID: iso.3.6.1.4.1.8072.3.2.10
% % iso.3.6.1.2.1.1.3.0 = Timeticks: (31099465) 3 days, 14:23:14.65
% % [...]
% % iso.3.6.1.2.1.1.5.0 = STRING: "avatar"
% % iso.3.6.1.2.1.25.1.4.0 = STRING: "root=/dev/xvda2 ro
% % root=/dev/xvda2 ro ip=:127.0.255.255::::eth0:dhcp"
% % \end{verbatim}
% % \Normalsize

\subsection{Know Your Systems}
Profile your application:
\begin{itemize}
	\item execution time (for example: {\tt time(1)})
	\item data sources and destination affect execution
	\item {\tt strace(1)} and friends for more detailed analysis
\end{itemize}

\addvspace{.5in}
Understand your system performance:
\begin{itemize}
	\item CPU load, memory (for example: {\tt top(1)}, {\tt vmstat(1)})
	\item disk I/O (for example: {\tt iostat(1)})
	\item user activity (for example: {\tt ac(1)}, {\tt lsof(8)}, {\tt sa(8)})
\end{itemize}

\subsection{Know Your Systems}
Network statistics:
\begin{itemize}
	\item ports and applications (for example: {\tt lsof(8)}, {\tt netstat(8)})
	\item packets in and out
	\item connection origin
	\item {\em NetFlow} etc.
\end{itemize}

\subsection{Context}
{\em Context} lets you find {\em relevant} events in
your haystack of metrics.

\begin{center}
	\includegraphics[scale=0.75]{pics/glass-needle.eps}
\end{center}

\subsection{No context.}
CPU load - 12 hours
\begin{center}
	\includegraphics[scale=0.9]{pics/cpu-12h.eps}
\end{center}

\subsection{No context.}
Disk I/O - 12 hours
\begin{center}
	\includegraphics[scale=0.9]{pics/disk-io-12h.eps}
\end{center}

\subsection{No context.}
Load Average - 12 hours
\begin{center}
	\includegraphics[scale=0.9]{pics/load-average-12h.eps}
\end{center}

\subsection{No context.}
Memory - 12 hours
\begin{center}
	\includegraphics[scale=0.9]{pics/memory-12h.eps}
\end{center}

\subsection{Some context.}
12 hours
\begin{center}
	\includegraphics[scale=0.36]{pics/cpu-12h.eps}
	\includegraphics[scale=0.36]{pics/disk-io-12h.eps} \\
	\includegraphics[scale=0.36]{pics/load-average-12h.eps}
	\includegraphics[scale=0.36]{pics/memory-12h.eps} \\
\end{center}

\subsection{With context.}
7 days
\begin{center}
	\includegraphics[scale=0.36]{pics/cpu-7day.eps}
	\includegraphics[scale=0.36]{pics/disk-io-7day.eps} \\
	\includegraphics[scale=0.36]{pics/load-average-7day.eps}
	\includegraphics[scale=0.36]{pics/memory-7day.eps} \\
\end{center}

\subsection{Know your systems.}
30 days
\begin{center}
	\includegraphics[scale=0.36]{pics/cpu-30day.eps}
	\includegraphics[scale=0.36]{pics/disk-io-30day.eps} \\
	\includegraphics[scale=0.36]{pics/load-average-30day.eps}
	\includegraphics[scale=0.36]{pics/memory-30day.eps} \\
\end{center}

\subsection{Turn {\em events} into {\em metrics.}}
\begin{itemize}
	\item Log it!
\end{itemize}

\addvspace{.5in}
\begin{itemize}
	\item Export counters/timers from within your application.
	\item Process logs and produce counters/timers:
\begin{verbatim}
awk '{print $9}' /var/log/httpd/access.log | sort | uniq -c
\end{verbatim}
	\item create a baseline
\end{itemize}

\addvspace{.5in}
\begin{itemize}
	\item Graph it. \\
	{\tt https://is.gd/tDCmQI}
\end{itemize}

\subsection{Monitoring/graphing}
SNMP based:
\begin{itemize}
	\item Cacti: \verb+http://www.cacti.net/+
	\item MRTG: \verb+http://oss.oetiker.ch/mrtg/+
	\item Observium: \verb+http://demo.observium.org/+
	\item ...
\end{itemize}
\vspace{.2in}
Other / complementary:
\begin{itemize}
	\item Ganglia: \verb+http://ganglia.info/+
	\item Munin: \verb+http://munin-monitoring.org/+
	\item Nagios: \verb+http://nagioscore.demos.nagios.com/+
	\item Graphite: \verb+http://graphite.wikidot.com/+
\end{itemize}
\vspace{.5in}
%More on monitoring and performance in a future lecture (if time permits).

\subsection{Context doesn't explain everything...}
...but it helps you look into what to investigate.

\begin{center}
	\includegraphics[scale=0.6]{pics/traffic-by-cipher.eps}
\end{center}

\subsection{Context doesn't explain everything...}
...but it helps you look into what to investigate.

\begin{center}
	\includegraphics[scale=0.6]{pics/traffic-by-site.eps}
\end{center}

\subsection{To the cloud!}
There’s a service for that. In the cloud. \\

Consider:
\begin{itemize}
	\item support / convenience vs. do-it-yourself
	\item integration with your other services
	\item data confidentiality
	\item data lock-in (esp. when trending data over years)
\end{itemize}

\subsection{Monitoring Pitfalls}
\vspace*{\fill}
\Huge
\begin{center}
Increasing the size of your haystack does not always
help in finding the needle.
\end{center}
\Normalsize
\vspace*{\fill}

\subsection{Monitoring Pitfalls}
\vspace*{\fill}
\Huge
\begin{center}
Increasing the size of your haystack does not always
help in finding the needle. \\
\vspace{.2in}
Email is not a scalable network monitoring solution.
\end{center}
\Normalsize
\vspace*{\fill}

\subsection{Monitoring Pitfalls}
\vspace*{\fill}
\Huge
\begin{center}
Increasing the size of your haystack does not always
help in finding the needle. \\
\vspace{.2in}
Email is not a scalable network monitoring solution. \\
\vspace{.2in}
Absence of a signal can itself be a signal.
\end{center}
\Normalsize
\vspace*{\fill}

\subsection{Monitoring Pitfalls}
\vspace*{\fill}
\Huge
\begin{center}
Increasing the size of your haystack does not always
help in finding the needle. \\
\vspace{.2in}
Email is not a scalable network monitoring solution. \\
\vspace{.2in}
Absence of a signal can itself be a signal. \\
\vspace{.2in}
Most of the value of your metrics only becomes evident
over time.
\end{center}
\Normalsize
\vspace*{\fill}

\subsection{Monitoring Pitfalls}
\vspace*{\fill}
\Huge
\begin{center}
Increasing the size of your haystack does not always
help in finding the needle. \\
\vspace{.2in}
Email is not a scalable network monitoring solution. \\
\vspace{.2in}
Absence of a signal can itself be a signal. \\
\vspace{.2in}
Most of the value of your metrics only becomes evident
over time. \\
\vspace{.2in}
This list is incomplete.
\end{center}
\Normalsize
\vspace*{\fill}

\newpage
\vspace*{\fill}
\begin{center}
	\Hugesize
		Hooray!\\ [1em]
	\hspace*{5mm}
	\blueline\\
	\hspace*{5mm}\\
		5 minute break
\end{center}
\vspace*{\fill}

\subsection{Team Missions}
Red Team: \verb+https://is.gd/LfrKPi+ \\

\vspace{.5in}

Blue Team: \verb+https://is.gd/kkXMQ2+ \\

\subsection{Entropy is the Enemy}
\vfill
\Huge
\begin{center}
The entropy of an isolated system never decreases.
\end{center}
\Normalsize
\vfill

\subsection{Entropy is the Enemy}
A static system is a useless system.
A useful system is being used.
\vspace{.5in}
\begin{itemize}
	\item data is processed; files are created, modified, removed
	\item software is added, upgraded, removed
	\item systems are created, copied, decommissioned
	\item instances / containers are even more short-lived,
		coming into existence and disappearing again as needed
\end{itemize}

\subsection{Single Systems are Fragile}
Individual systems created and configured by hand are
fragile.  Our processes need to be repeatable,
automated, reliable. \\

Recall previous lectures:

\begin{itemize}
	\item OS installation
	\item package management
	\item multi-user basics
	\item automation
	\item recovery / restores
\end{itemize}

\subsection{Reproducable}
\vspace*{\fill}
\begin{center}
	\includegraphics[scale=0.3]{pics/throw.eps} \\
	\vspace*{\fill}
	{\em ``Never trust a computer you can't throw out the
window.''} -- Woz
\end{center}

\subsection{Evolution of Configuration Management}
``I set up a server over here to do X.  Replicate that
setup on all the others.'' \\

\subsection{Evolution of Configuration Management}
``I set up a server over here to do X.  Replicate that
setup on all the others.'' \\

``I know how to do this!  Watch me!'' \\
\begin{verbatim}
$ ssh root@server1
# rsync -e ssh -avz / server2:/
\end{verbatim}
\vspace{.5in}
``{\tt /etc}?  What's that?''

\subsection{Evolution of Configuration Management}
\\

\vspace{.5in}
\Huge
        \begin{tabular}{ l | l | l }
        & shareable content & unshareable content \\
        \hline
        static data & {\tt /usr} & {\tt /boot} \\
        & {\tt /opt} & {\tt /etc} \\
        \hline
        variable data & {\tt /home} & {\tt /tmp} \\
        & {\tt /var/mail} & {\tt /var/run} \\
        \hline
        \end{tabular}
\Normalsize

\subsection{Every Sysadmin ever...}
\begin{enumerate}
	\item {\tt scp(1)}
	\item {\tt rsync(1)}
	\item some sort of parallel {\tt ssh(1)} of the above
	\item switch to {\em pull}
	\item add mutual authentication
	\item but effectively ignore mismatches, because doing things the right way is difficult and inconvenient
	\item switch to {\em push} with remote d\ae mon
	\item write an inventory database
	\item deploy a well-known CM system
\end{enumerate}
\vspace{.25in}

Finally: find something it can't do, goto 1.

\subsection{Base configuration vs. service definition}
Your servers have {\em unique}, yet predictable
properties.  E.g.:

\begin{itemize}
	\item network configuration
	\item critical services: DNS, NTP, Syslog 
	\item minimum OS / software version
	\item user management
	\item common service configuration (e.g. {\tt sshd(8)})
	\item ...
\end{itemize}

\subsection{Base configuration vs. service definition}
Different sets of servers have {\em shared}
properties.  For example, consider an HTTP server:

\begin{itemize}
	\item minimum server software
	\item appropriate TLS specification
	\item shared TLS certificate and key
	\item database configuration
	\item static content (HTML / JS / CSS files)
	\item ...
\end{itemize}

\subsection{Pets vs. Cattle}
``Pets'':
\begin{itemize}
	\item unique, cheerful hostnames
	\item single systems grown over time, lovingly configured by hand
	\item when sick, everybody is very concerned
	\item slowly nursed back to life
\end{itemize}
\vspace{.25in}
``Cattle'':
\begin{itemize}
	\item predictable, boring hostnames
	\item almost identical to all others
	\item centrally managed, easy to recreate
	\item when sick, they get taken out back and shot
	\item quickly replaced by another
\end{itemize}

\subsection{Service definitions}
\small
\begin{verbatim}
class syslog {
  include cron
  include logrotate
  package {
    'syslog−ng' :
      ensure  => latest ,
      require => Service['syslog−ng'];
  }
  service {
    'syslog−ng' :
      ensure => running ,
      enable => true;
  }
  file {
    '/etc/syslog−ng/syslog−ng.conf':
      ensure  => file,
      source  => 'puppet:///syslog/syslog−ng.conf',
      mode    => '0644',
      owner   => 'root',
      group   => 'root',
      require => Package['syslog-ng'],
      notify  => Service['syslog-ng'];

    '/etc/logrotate.d/syslog-ng':
      ensure  => file,
      source  => 'puppet:///syslog/logrotate-syslog−ng',
      mode    => '0644',
      owner   => 'root',
      group   => 'root',
      require => Package['logrotate'];
  }
}
\end{verbatim}
\Normalsize

\subsection{Service definitions}
\smallish
\begin{verbatim}
package "ldap-utils" do
  action :upgrade
end

template "/etc/ldap.conf" do
  source "ldap.conf.erb"
  mode   00644
  owner  "root"
  group  "root"
end

%w{ account auth password session }.each do |pam|
  cookbook_file "/etc/pam.d/common-#{pam}" do
    source   "common-#{pam}"
    mode     00644
    owner    "root"
    group    "root"
    notifies :restart, resources(:service => "ssh"), :delayed
  end
end
\end{verbatim}
\Normalsize

\subsection{Service definitions}
\smallish
\begin{verbatim}
bundle agent sshd(parameter) {
    files:
        "/tmp/sshd_config.tmpl"
            perms     => mog("0600","root","root"),
            copy_from => secure_cp("/templates/etc/ssh/sshd_config",
                                   "cf-master.example.com");

        "/etc/ssh/sshd_config"
            perms     => mog("0600","root","root"),
            create    => true,
            edit_line => expand_template("/tmp/sshd_config.tmpl"),
            classes   => if_repaired("restart_sshd");

    commands:
        restart_sshd::
            "/etc/rc.d/sshd restart"
}
\end{verbatim}
\Normalsize

\subsection{Team Missions}

Black Team: \verb+https://is.gd/zQFtGJ+ and
\verb+https://is.gd/b1fN36+ \\

\vspace{.5in}

Green Team: \verb+https://is.gd/mWKosu+ and
\verb+https://is.gd/ejJT1T+ \\


\subsection{CM Requirements}
\begin{itemize}
	\item software installation
\end{itemize}

\subsection{CM Requirements}
\begin{itemize}
	\item software installation
	\item service management / supervising
\end{itemize}

\subsection{CM Requirements}
\begin{itemize}
	\item software installation
	\item service management / supervising
	\item file permissions / ownership
\end{itemize}

\subsection{CM Requirements}
\begin{itemize}
	\item software installation
	\item service management / supervising
	\item file permissions / ownership
	\item static files
\end{itemize}

\subsection{CM Requirements}
\begin{itemize}
	\item software installation
	\item service management / supervising
	\item file permissions / ownership
	\item static files
	\item host-specific data
\end{itemize}

\subsection{CM Requirements}
\begin{itemize}
	\item software installation
	\item service management / supervising
	\item file permissions / ownership
	\item static files
	\item host-specific data
\end{itemize}
\vspace{.25in}
\begin{itemize}
	\item command-execution
\end{itemize}

\subsection{CM Requirements}
\begin{itemize}
	\item software installation
	\item service management / supervising
	\item file permissions / ownership
	\item static files
	\item host-specific data
\end{itemize}
\vspace{.25in}
\begin{itemize}
	\item command-execution
	\item data collection
\end{itemize}

\subsection{One more layer of abstraction...}
\vspace{.5in}

The objective of a CM system is not to {\em make
changes} on a system. \\

\vspace{.5in}

The objective of a CM system is to {\em assert state}.

\subsection{CM States}
\vspace*{\fill}
\begin{center}
	\includegraphics[scale=0.7]{pics/host-states.eps} \\
\end{center}
\vspace*{\fill}

\subsection{Circles around things}
Group your resources into {\em sets}. \\
\vspace{.5in}

\begin{itemize}
	\item functional groupings
	\item services
	\item users
	\item hosts
\end{itemize}

\subsection{Circles around things}
\vspace*{\fill}
\begin{center}
	\includegraphics[scale=0.7]{pics/change-sets.eps} \\
\end{center}
\vspace*{\fill}

\subsection{Circles around things}
\vspace*{\fill}
\begin{center}
	\includegraphics[scale=0.7]{pics/groups-machines.eps} \\
\end{center}
\vspace*{\fill}

\subsection{Circles around things}
\vspace*{\fill}
\begin{center}
	\includegraphics[scale=1.0]{pics/host-sets.eps} \\
\end{center}
\vspace*{\fill}

\subsection{CMs configure complex systems}
CM systems are complex themselves. \\
\vspace{.25in}

CM systems are inherently trusted. \\
\vspace{.25in}

CM systems can break everything.  To the degree that
you can't unbreak things afterwards. \\
\vspace{.5in}

Consider:
\begin{itemize}
	\item staged rollout of change sets
	\item automated error detection and rollback
	\item self-healing properties
	\item authentication and privilege
\end{itemize}


\subsection{Idempotence}
CM systems assert state.  For this, all operations
must be {\em idempotent}. \\
\vspace{.5in}

\begin{displaymath}
f(f(x)) \equiv f(x)
\end{displaymath}

\begin{displaymath}
| |-1| | \equiv |-1|
\end{displaymath}

\subsection{Idempotence}
CM systems assert state.  For this, all operations
must be {\em idempotent}. \\
\vspace{.5in}

\begin{displaymath}
f(f(x)) \equiv f(x)
\end{displaymath}

\begin{displaymath}
| |-1| | \equiv |-1|
\end{displaymath}

\begin{verbatim}
$ rm resolv.conf
\end{verbatim}

\subsection{Idempotence}
CM systems assert state.  For this, all operations
must be {\em idempotent}. \\
\vspace{.5in}

\begin{displaymath}
f(f(x)) \equiv f(x)
\end{displaymath}

\begin{displaymath}
| |-1| | \equiv |-1|
\end{displaymath}

\begin{verbatim}
$ rm resolv.conf                                    # idempotent
$ echo "nameserver 192.168.0.1" > resolv.conf
\end{verbatim}

\subsection{Idempotence}
CM systems assert state.  For this, all operations
must be {\em idempotent}. \\
\vspace{.5in}

\begin{displaymath}
f(f(x)) \equiv f(x)
\end{displaymath}

\begin{displaymath}
| |-1| | \equiv |-1|
\end{displaymath}

\begin{verbatim}
$ rm resolv.conf                                    # idempotent
$ echo "nameserver 192.168.0.1" > resolv.conf       # idempotent
$ echo "nameserver 192.168.0.2" >> resolv.conf
\end{verbatim}

\subsection{Idempotence}
CM systems assert state.  For this, all operations
must be {\em idempotent}. \\
\vspace{.5in}

\begin{displaymath}
f(f(x)) \equiv f(x)
\end{displaymath}

\begin{displaymath}
| |-1| | \equiv |-1|
\end{displaymath}

\begin{verbatim}
$ rm resolv.conf                                    # idempotent
$ echo "nameserver 192.168.0.1" > resolv.conf       # idempotent
$ echo "nameserver 192.168.0.2" >> resolv.conf      # not idempotent
$ chown root:wheel resolv.conf
\end{verbatim}

\subsection{Idempotence}
CM systems assert state.  For this, all operations
must be {\em idempotent}. \\
\vspace{.5in}

\begin{displaymath}
f(f(x)) \equiv f(x)
\end{displaymath}

\begin{displaymath}
| |-1| | \equiv |-1|
\end{displaymath}

\begin{verbatim}
$ rm resolv.conf                                    # idempotent
$ echo "nameserver 192.168.0.1" > resolv.conf       # idempotent
$ echo "nameserver 192.168.0.2" >> resolv.conf      # not idempotent
$ chown root:wheel resolv.conf                      # idempotent
$ chmod 0644 resolv.conf
\end{verbatim}

\subsection{Idempotence}
CM systems assert state.  For this, all operations
must be {\em idempotent}. \\
\vspace{.5in}

\begin{displaymath}
f(f(x)) \equiv f(x)
\end{displaymath}

\begin{displaymath}
| |-1| | \equiv |-1|
\end{displaymath}

\begin{verbatim}
$ rm resolv.conf                                    # idempotent
$ echo "nameserver 192.168.0.1" > resolv.conf       # idempotent
$ echo "nameserver 192.168.0.2" >> resolv.conf      # not idempotent
$ chown root:wheel resolv.conf                      # idempotent
$ chmod 0644 resolv.conf                            # idempotent
$ yum install frozzle
\end{verbatim}


\subsection{Idempotence}
CM systems assert state.  For this, all operations
must be {\em idempotent}. \\
\vspace{.5in}

\begin{displaymath}
f(f(x)) \equiv f(x)
\end{displaymath}

\begin{displaymath}
| |-1| | \equiv |-1|
\end{displaymath}

\begin{verbatim}
$ rm resolv.conf                                    # idempotent
$ echo "nameserver 192.168.0.1" > resolv.conf       # idempotent
$ echo "nameserver 192.168.0.2" >> resolv.conf      # not idempotent
$ chown root:wheel resolv.conf                      # idempotent
$ chmod 0644 resolv.conf                            # idempotent
$ yum install frozzle                               # not idempotent
$ yum install frozzle-1.2.3
\end{verbatim}

\subsection{Idempotence}
CM systems assert state.  For this, all operations
must be {\em idempotent}. \\
\vspace{.5in}

\begin{displaymath}
f(f(x)) \equiv f(x)
\end{displaymath}

\begin{displaymath}
| |-1| | \equiv |-1|
\end{displaymath}

\begin{verbatim}
$ rm resolv.conf                                    # idempotent
$ echo "nameserver 192.168.0.1" > resolv.conf       # idempotent
$ echo "nameserver 192.168.0.2" >> resolv.conf      # not idempotent
$ chown root:wheel resolv.conf                      # idempotent
$ chmod 0644 resolv.conf                            # idempotent
$ yum install frozzle                               # not idempotent
$ yum install frozzle-1.2.3                         # "it depends"
\end{verbatim}

\subsection{Convergence and Eventual Consistency}

Note: while idempotence enables self-healing and may
allow you to not keep state, it does not guarantee
efficiency! \\

\vspace{.5in}

CM systems should ensure changes are:
\begin{enumerate}
	\item idempotent (well, that part's on you)
	\item only applied if needed
	\item eventually consistent
\end{enumerate}
\vspace{.5in}

This often requires complexity (oh no!), coordination
with and awareness of other systems.  {\em Service
Orchestration} has developed as a separate, related
discipline to help address this.

\subsection{Distributed Systems}
CM systems are {\em distributed} systems.  As such,
they are subject to the CAP Theorem: \\

{\em Consistency}: all systems managed by the CM are
consistent within their respective service definition.
\\

{\em Availability}: the services managed by the CM are
kept available, even if no further updates or change
sets can be retrieved. \\

{\em Partition tolerance}: the CM system can (continue
to) operate despite interruptions between its
components; e.g. intermediate (coordinated) changes
are not required.

\subsection{Configuration Management Overlap}

Your configuration management system provides or
enables:

\begin{itemize}
	\item a remote command execution agent
	\item a reporting agent
	\item a reporting infrastructure
	\item role-based actions and visibility
\end{itemize}
\vspace{.5in}

The same principles enabling reliable configuration
management can thus also be used for information
security related tasks:
\begin{itemize}
	\item detection of deviation of known state
	\item integrity checks and intrusion detection
	\item patch management
	\item automated quarantine
\end{itemize}

\subsection{Configuration Management Overlap}
Configuration Management overlaps with numerous other
areas:

\begin{itemize}
	\item backup (expendable systems, data classiciation, ...)
	\item software deployment (base OS, application packages, ..)
	\item monitoring (central reporting and ad-hoc data collection, ...)
	\item revision control and audit logs (CM changes are code changes!)
	\item compliance enforcement (e.g., baseline configurations)
	\item ...
\end{itemize}

\subsection{Overlap with other systems}
\vspace*{\fill}
\begin{center}
	\includegraphics[scale=0.6]{pics/cm-overlap.eps} \\
\end{center}
\vspace*{\fill}

\subsection{More than just servers...}
Configuration Management is not just for servers.  You
also need to manage configurations for:
\vspace{.25in}

\begin{itemize}
	\item desktops and item mobile clients
	\item network equipment
	\item load balancers
	\item containers
	\item ...
\end{itemize}

\subsection{Configuration Management Impact}
Think scale!

\begin{center}
	\includegraphics[scale=1.2]{pics/shoot-yourself-in-the-foot.eps} \\
\end{center}


\subsection{Reading}

Additional topics to research:
\begin{itemize}
	\item Service Orchestration
	\item Continuous Deployment / Continuous Integration
	\item Infrastructure as Code
	\item Information Technology Infrastructure Library (ITIL)
\end{itemize}
\vspace{.25in}

Relevant links:
\begin{itemize}
	\item {\tt http://www.infrastructures.org/bootstrap/recovery.shtml}
	\item {\tt https://is.gd/paZ7qu}
	\item {\tt https://www.engineyard.com/blog/pets-vs-cattle}
	\item {\tt http://markburgess.org/blog\_cap.html}
	\item {\tt http://markburgess.org/blog\_cap2.html}
	\item {\tt https://aws.amazon.com/opsworks/chefautomate/}
	\item {\tt https://puppet.com/product/managed-technology/aws}
\end{itemize}

\end{document}
